%! Author = simon
%! Date = 04-11-2023

% Preamble
\documentclass[11pt]{article}
\title{Existing solutions}


% Packages
\usepackage{amsmath}

% Document
\begin{document}
    \maketitle
    \begin{abstract}
        In this section, we will discuss extant solutions and their operational mechanisms pertaining to the provided
        problem statement.
        We will analyze prominent software applications in the market,
        including Rejseplanen,Google Maps, and Apple Maps.
    \end{abstract}


    \section{Rejseplanen}\label{sec:rejseplanen}

    \textbf{What is Rejseplanen?}

    The ``Rejseplanen'' is an online service used in Denmark to plan journeys using public transport.
    It is used to find information about routes, timings, prices and ticket purchases for trains, buses, metro, trams and ferries
    throughout the country. \newline

    With ``Rejseplanen'', you can input your departure point and destination point to get information about the fastest,
    cheapest or most convenient way to travel from point A to B using public transport.
    The service provides detailed information about the departure times, transfers, expected arrival times and prices.
    It also provides any additional information about the delays or canceled departures in the operation of public transport.\newline

    \subsection{Data Sources for Rejseplanen}\label{subsec:where-does-rejseplanen-get-their-data-from?}

    Rejseplanen is jointly owned by DSB, Movia, Metroselskabet, NT, Midttrafik, Sydtrafik, Fynbus, and BAT.
    As a result of its ownership by these various transportation service providers, Rejseplanen has access to the
    data necessary to determine the most efficient public transport routes.

    \subsection{What makes Rejseplanen great?}\label{subsec:what-makes-rejseplanen-great?}
    \begin{description}
        \item [Comprehensive Data Integration] - Rejseplanen integrates data from various transportation providers, thus ensuring up-to-date and accurate information
        \item [Very efficient] - Rejseplanen provides a quick and convenient travel plan, saving time for daily commuters
        \item [User friendly] - Rejseplanen provides additional features such as filtering out specific trains or buses, so your route matches your personal preferences and enhancing the user experience
    \end{description}


    \section{Google Maps / Apple Maps}\label{sec:google-maps-/-apple-maps}
    Man what hell is this


\end{document}
