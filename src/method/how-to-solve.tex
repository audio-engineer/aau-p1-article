\section{How can we help solve this problem?}\label{sec:how-can-we-help-solve-this-problem?}

The goal of this project is to help people understand their travel options, and also help them choose the one most
inline with their preferences.
Making this program would also allow users to plan their journey days before, which would improve traffic flow
overall and quality of life for the consumers.

We gather data from ``rejseplanens'' API such that we can request data about the public transport in Denmark.
We further investigate the environmental factors where we investigate how much \unit{CO_{2}} emission
different car types and public transport types causes per kilometer.

We research how much \unit{CO_{2}} emission different types of cars cause and set them to an average for
the program parameters to make it simpler to calculate routes with less \unit{CO_{2}} emission.
The same goes for public transport.
We research how much \unit{CO_{2}} emission trains, buses and metros emit.

We allow users to make different priorities such as the time of arrival, cost, environmental impact and comfort.
Furthermore, a filter is added where the user is able to set their own priorities.

The user interface (UI) is supposed to be user-friendly, such that it can be used effectively by all
consumers.
The UI we are using for this project is the terminal, which is text-based.
The reason for the choice of our UI is limited by the projects restrictions in which we were recommended to make a
terminal UI instead of another more user-friendly graphical UI\@.
