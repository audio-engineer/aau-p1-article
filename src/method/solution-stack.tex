\section{Solution stack}\label{sec:solution-stack}

In the following section, we will take a closer look at the solution stack that the development team has chosen to use
to create the software that will constitute the problem solution.

\subsection{Development team}\label{subsec:development-team}

The development team consisted of seven young computer science students with different levels of previous development
experience.

\subsection{Basic requirements}\label{subsec:basic-requirements}

Before the development team began the development process, a wide variety of basic technological and workflow-related
requirements and constraints were discussed.

The team understood very quickly that high productivity and development efficiency were one of the top priorities for
the task at hand, due to the relatively short deadline, development team size and the complexity of the software's
logic.

Other constraints were identified as well, such as the fact that while every member of the development team had
different levels of experience with different programming languages and technologies, everyone was familiar with the
C programming language.

Finally, coming from a wide range of backgrounds and having different preferences, the individual members of the team
wanted to be able to make individual choices regarding their development setups, while still maintaining project-level
coherency.
Examples of such individual preferences included which development operating system each member would work on (such as
Windows 11, macOS, various Linux distributions, and also Windows Subsystem for Linux (WSL)).
Additionally, individual preferences extended to choices regarding what code editor or intelligent development
environment (IDE) the member would use.
And finally, another constraint dictated by personal preferences was what tools were available for the chosen
development platform, such as compilers, linters, etc.

Those three basic realizations largely dictated all the other choices of technologies, workflows and other development
tools.

\subsection{Version control}\label{subsec:version-control}

When dealing with a software project of this size and complexity, combined with a development team of this size, it is
crucial to be able to collaborate remotely over the internet.
Also, it is paramount that the team is able to trace incremental changes in the software source code.
This is where version control and version control systems come into play.

\subsubsection{Version control system}

For this project, the team selected the \lstinline{git} version control system, which is modern, efficient and
relatively easy to use.
The usage of \lstinline{git} allowed each team member to develop incremental features and changes on their own
development machine and then combine the changes on the \lstinline{main} version control branch.
To further ease the process and make individual development fully independent, a server was needed to which the
individual members' contributions could be uploaded.
For this task, \href{https://github.com/}{GitHub} was chosen, and a public code repository was created using the private
GitHub account of one of the team members.
The repository can be found here: \href{https://github.com/audio-engineer/aau-p1-software}{aau-p1-software}.

\subsubsection{Branching model}

Modern version control using \lstinline{git} relies on the idea of \textit{branching}.
\textit{Branches} consist of a series of \textit{commits}, which are snapshots of the source code taken at arbitrary
times by the developers while working on their development machines.
The team decided that the easiest approach to branching would be a model based heavily on \textit{trunk-based
development}~\cite{trunk-based}, which is a simple model where all changes from individual branches are merged directly
into the project's \lstinline{main} branch without additional branching complexity.
Additionally, to keep the development routines as simple and efficient as possible, it was decided that each developer
would have their own branch on which they would continuously commit their individual contributions.
Those branches were then continuously uploaded (\textit{pushed} in \lstinline{git} terminology) to GitHub, where they
were merged into the \lstinline{main} branch.
Each time after a merge was completed, the contributor in question's branch was reset to match the \lstinline{main}
branch, upon which the development cycle was resumed.

We will take a closer look at how it was implemented practically in the Section~\ref{sec:workflows-and-routines}.

\subsection{Development systems and tools}\label{subsec:development-systems-and-tools}

As previously mentioned, the teams' personal development setup preferences varied, not least in the choice of code
editor or IDE, and development machine operating system.
Simultaneously, it was determined that to maximize efficiency, some project configurations and other related settings
would have to be shared across development machines and operating systems.
This was deemed necessary to prevent that each developer would have to set up their local development environment
from scratch.
To mitigate this issue, some sort of configuration sharing was necessary to be introduced into the development flow.
Luckily, a majority of the developers were already familiar with the IDE vendor JetBrain's products, which offer great
support for shared configurations~\cite{shared-config}.
Therefore, a shared configuration was added to the source code repository, which was then read by each developer's
IDE upon opening the project.
Examples of configurations that were shared include code indentation and bracket placement style, unit test runner
configuration, text encoding, required IDE plugins and code linting settings.

\subsection{Programming languages}\label{subsec:programming-languages}

As was mentioned in the Section~\ref{subsec:basic-requirements}, the C programming language was the single
programming language that all developers in the team were familiar with, which is one of the primary reasons why it was
chosen as the development language.

Another reason was the flexibility that C offers: It can be developed and compiled for a wide range of target systems.
In other words, C offers cross-platform compatibility both from development and distribution perspectives.

It was also decided early on that the solution application would be a console application, and C also has the advantage
that it is an excellent language for console application development.

Finally, C has been on the market for many decades and is a very well-established language with exceptional tooling and
availability of libraries, compilers, IDEs, and frameworks that support it, making it a good choice for many tasks.

But while C made up the majority of the source code, it also turned out to be necessary to introduce C++ into the
project, due to the usage of the \textit{GoogleTest} unit testing library.
GoogleTest is originally built for usage with C++, but can easily be adopted to be used with purely C-based projects,
only requiring slight modifications to how the project is built and how the tests are written.
Also, luckily GoogleTest is almost exclusively built on macros, which made writing tests a lot easier for the team,
since only very little or almost no C++ syntax had to be learned.
This will be further discussed in the Section~\ref{subsec:testing}.

\subsection{Integrations}\label{subsec:integrations}

// REST service, Rejseplanen API

\subsection{Build system}\label{subsec:build-system}

One of the disadvantages of C, though, is that it requires a solid and substantial ecosystem of build tools and
compilers around the C source code to be able to build the distribution-ready application.
One approach would be to let each developer set up their own build system to be able to compile the source code on their
local development environment, and then later decide on a \textit{master} compilation machine where the distribution
version would be compiled.
But the development team chose an approach where the build configuration was built into the source code itself, making
it easy to share it across local development environments.
This also allowed each developer to build the software using the same build parameters.

The primary tool picked to solve this task was CMake, which is a build automation tool that generates build
configuration files based on a common project configuration.

CMake also has the advantage that it can prepare the compilation machine for compilation by searching for compilation
dependencies aka ``dependencies of dependencies'' without which the compilation would fail.
For this purpose, the team made use of the \lstinline{pkg-config} CMake integration. \lstinline{Pkg-config} is a tool
that searches the compilation machine exactly for those prerequisite dependencies to make sure that all direct
dependencies and if they are not found, the compilation is aborted.
If they are, on the other hand, found, the application compilation is resumed and completed.

% TODO Insert CMake dependency, compilation and linkage flowchart

Furthermore, CMake also allows for the addition and management of external software dependencies such as libraries
and frameworks, which will be discussed in the Section~\ref{subsec:libraries-and-frameworks}.

\subsection{Libraries and frameworks}\label{subsec:libraries-and-frameworks}

% TODO Add more content to this subsection
// cURL, cJSON, OpenSSL

\subsection{Code style and quality}\label{subsec:code-style-and-quality}

When working on larger scale software projects where more than one developer is involved, it is also crucial to keep the
source code readable, maintainable and coherent to be able to easily develop and refactor it later.
For this reason, the team decided to use a common code style and use tools to ensure that this style was complied with.
As a baseline, it was decided that the Google C++ Style Guide~\cite{google-style} would be used for basic stylistic
features such as function, data and file naming conventions, an indentation width of two spaces, bracket placement, etc.
While the Google C++ Style Guide is, as the name suggests, made for C++ projects, it can easily be adopted for usage
with C projects by simply adhering only to the style guidelines that regard features that both C++ and C share.

To assist the developer team with adhering to those guidelines, the \textit{Clang-Format}~\cite{clang-format} and
\textit{Clang-Tidy}~\cite{clang-tidy} tools provided by the \textit{Clang} compiler were used to automatically lint
the code for breaches of style.
This process was made even easier due to the seamless integration of these tools into the JetBrains CLion IDE that was
used by a majority of developers, which made it possible to lint the source code while writing it.

ClangFormat, ClangTidy, EditorConfig, Google C++ Style Guide, pull requests and code reviews.
We will take a closer look at how it was implemented practically in the next Section~\ref{sec:workflows-and-routines}.

% textidote: ignore begin

\subsection{Testing}\label{subsec:testing}

% TODO Add more content to this subsection
// Unit tests, GoogleTest

\subsection{CI/CD}\label{subsec:ci/cd}

% TODO Add more content to this subsection
// GitHub Workflows/Actions

\subsection{Target systems}\label{subsec:target-systems}
% textidote: ignore end

% TODO Add more content to this subsection
// Cross-platform (Windows, macOS, Linux)
