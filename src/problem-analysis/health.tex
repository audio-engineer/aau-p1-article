\section{Health consequences}\label{sec:health-consequences}

Another aspect of the commute is the health benefits and consequences of different types of transportation.
The long commute has other health effects than physical ones - it also affects relationships.

Long commuting couples has a higher risk of separation compared to shorter commuting ones~\cite{sandow2011}.
This makes sense, as commuting takes up valuable time couples could have been spending together, which in turn could
have strengthened their relationships.
Along with the increased rates of stress-related symptoms when commuting, couples have a harder time staying together
and focusing on each other rather than on the contents of the commute.

Even more concerning is that men in couples that commute on a temporary basis experience the highest risk of separation.
Also, women experience less risk of separation when commuting for a longer time period.
These risks also depend on where the couple resides~\cite{sandow2011}.
These separation risk theories take into consideration that the commute has to be at least 30 kilometers long.
This also makes sense, as the sporadic nature of switching one's habits, such as commuting patterns, in one's household
can have a negative effect in the same way other sporadic events can severely change the dynamic of the relationship,
such as having a baby, losing a family member or getting fired.
Therefore, it is interesting for a user of our program to be able to select a maximum commuting length, as this will
probably indirectly benefit them in their relationships.



