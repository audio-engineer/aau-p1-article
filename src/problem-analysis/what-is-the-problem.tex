\section{What is the problem?}\label{sec:what-is-the-problem?}

Commuting is a ubiquitous and integral part of modern life, affecting millions of people daily as they travel between
their residences and workplaces, educational institutions, or other destinations.
The objective is to delve into the issue of commuting, encompassing the issue's existence, its potential consequences,
the current perceptions of commuting, and the necessity to define the subject and purpose of addressing this issue.
A crucial facet of this analysis is to illuminate the concealed costs associated with commuting, including time, money,
and its impact on the climate, empowering individuals to make informed travel decisions~\cite{alma9921355859805762}.
% TODO Find `Enda, 2011` source and append it to \cite after `alma9921355859805762`

Commuting is a universal challenge that underpins the daily lives of individuals across the globe.
The existence of this issue is deeply rooted in the challenges faced by commuters as they navigate various modes of
transportation, from personal vehicles to public transit.
This existence is compounded by the prevalent issues of traffic congestion, stress, time inefficiency, heightened
greenhouse gas emissions, and financial strain.
These problems are accentuated by the continued urbanization and centralization of job opportunities.
As our cities expand, more individuals are drawn to urban centers in search of employment opportunities, resulting in a
growing commuting dilemma.
% TODO Find `Enda, 2011` source and insert it after `dilemma` and before `.`
The consequences of unaddressed commuting issues are multifaceted and profoundly impactful, affecting individuals,
society, and the environment.
Traffic congestion, particularly in urban areas during peak hours, results in reduced productivity, increased fuel
consumption, and growing frustration among commuters.
Extended commutes also lead to stress, anxiety, and various health-related issues, including cardiovascular ailments,
adding to the health burden.
The time spent in daily commutes consumes significant portions of daily life, diverting valuable time from work,
leisure, and personal activities.
Moreover, the reliance on personal vehicles intensifies air pollution and contributes to the pressing challenge of
climate change, raising ecological concerns.
The financial burden of commuting, encompassing expenses such as fuel, vehicle maintenance, public transportation fares,
and parking fees, places individuals and families under considerable financial strain~\cite{alma9921355859805762}.
The contemporary perspective on commuting portrays it as an indispensable yet frequently challenging facet of urban
life.
Individuals, including university students, often navigate their daily commutes through a combination of personal cars,
public transportation, cycling, and walking.
This experience is characterized by crowded public transit, traffic congestion, and a perpetual struggle to manage time
effectively.
The daily rigors of commuting frequently lead to physical and mental exhaustion, even in the face of technological
advancements and urban planning.
% TODO Find `Enda, 2011` source and insert it after `planning` and before `.`
If we focus on the issue of commuting, particularly in urban and suburban settings, as well as its implications for
university students.
It encompasses various transportation modes, including personal vehicles, public transit, cycling, and walking.
Additionally, it scrutinizes the societal and individual impacts of commuting, encompassing both positive and negative
aspects.
This subject holds relevance for students who must manage their academic pursuits alongside their daily commutes~\cite{alma9921355859805762}.
The purpose of this analysis is to raise awareness about the profound impact of commuting on individuals, society, and
the environment, with a specific emphasis on the implications for students.
By shedding light on the concealed costs of commuting, which include time, money, and climate effects, the analysis aims
to empower students and individuals to make well-informed decisions regarding their daily travel routines.
Furthermore, it underscores the necessity for innovative solutions and policy changes to mitigate negative consequences
and promote sustainable and efficient transportation options that can make the lives of university students and working
professionals more manageable.
% TODO Find `Enda, 2011` source and insert it after `manageable` and before `.`
Often the average individual, frequently underestimate the concealed costs associated with their daily journeys:

\begin{itemize}
    \item \textbf{Time}: Commuting consumes a substantial portion of daily life that could be more productively
    allocated to work, leisure, or personal activities, thus influencing students' academic and personal lives.
    \item \textbf{Money}: The expenses associated with commuting, including fuel, maintenance, parking fees, and public
    transportation fares, accumulate over time and impact individuals' financial stability, an issue that particularly
    affects students managing their budgets.
    \item \textbf{Climate Impact}: Commuting contributes to carbon emissions and air pollution, exacerbating
    environmental challenges, and underscores the significance of environmentally conscious transportation choices that
    resonate with students~\cite{alma9921355859805762}.
\end{itemize}

Commuting is a pervasive issue that significantly influences the daily lives of people, particularly university
students.
The consequences of unaddressed commuting problems encompass physical and mental health challenges, financial burdens,
and environmental impacts.
Current perceptions of commuting highlight its necessity but also the challenges it poses.
By making individuals aware of the hidden costs of commuting, such as time, money, and its environmental impact, this
analysis aims to empower individuals, especially students, to make more informed travel decisions.
Moreover, it emphasizes the need for innovative solutions and policy changes to mitigate negative consequences and
promote sustainable and efficient transportation options.
In doing so, it seeks to improve the quality of life for students and the broader population while addressing pressing
environmental concerns.
