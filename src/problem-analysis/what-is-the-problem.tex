\section{Shortcomings of commuting}\label{sec:shortcomings-of-commuting}

This Section will dive into the issue of commuting, encompassing the issue's existence, its potential consequences,
the current perceptions of commuting, and the necessity to define the subject and purpose of addressing this issue.
A crucial facet of this analysis is to shed a light on the concealed costs associated with commuting, which is time,
money, climate and health impact, empowering individuals to make informed travel decisions~\cite{alma9921355859805762}.
% TODO Find `Enda, 2011` source and append it to \cite after `alma9921355859805762` (?)

Commuting is a universal challenge that individuals across the globe face daily.
The existence of this challenge is deeply rooted in the challenges faced by commuters as they navigate various modes of
transportation, from personal vehicles to public transit.
This existence is compounded by the prevalent issues of traffic congestion, stress, time inefficiency, heightened
greenhouse gas emissions, and financial strain.
These issues are accentuated by the continued urbanization and centralization of job opportunities.
As our cities expand, more individuals are drawn to urban cities in search of employment opportunities, resulting in a
growing commuting dilemma.
% TODO Find `Enda, 2011` source and insert it after `dilemma` and before `.` (?)

The consequences of the problem remaining unaddressed will have a profound impactful, affecting individuals, society,
and the environment.
Traffic congestion, particularly in urban areas during peak hours, results in reduced productivity, increased fuel
consumption, and growing frustration among commuters.
Extended commutes also lead to stress, anxiety, and various health-related issues.
The time spent in daily commutes consumes significant portions of daily life, diverting valuable time from work,
rest, and personal life.
Moreover, the reliance on diesel cars intensifies air pollution and contributes to the pressing challenge of
climate change, raising ecological concerns.
The financial burden of commuting, encompassing expenses such as fuel, vehicle maintenance, public transportation fares,
and parking fees, places individuals and families under considerable financial strain~\cite{alma9921355859805762}.
A wide range of individuals, such as kids, teen, adults and elders, often navigate their daily commutes through a
combination of personal cars, public transportation, cycling, and walking.
This experience is characterized by crowded public transit, traffic congestion, and a perpetual struggle to manage time
effectively.
The daily challenges of commuting frequently lead to physical and mental exhaustion, even in the face of urban growth.
% TODO Find `Enda, 2011` source and insert it after `planning` and before `.` (?)
It encompasses various transportation modes, including personal vehicles, public transit, cycling, and walking.
Additionally, it scrutinizes the societal and individual impacts of commuting, encompassing both positive and negative
aspects.
This subject holds relevance for people who must manage their work or academic pursuits alongside their daily
commutes~\cite{alma9921355859805762}.

The purpose of this analysis is to raise awareness about the profound impact of commuting on individuals, society, and
the environment.
By shedding light on the concealed costs of commuting, which include time, money, and climate and health effects, the
analysis aims to empower individuals to make well-informed decisions regarding their daily travel routines.
Furthermore, it underscores the necessity for innovative solutions and policy changes to mitigate negative consequences
and promote sustainable and efficient transportation options that can make the lives of students and working
professionals more manageable.
% TODO Find `Enda, 2011` source and insert it after `manageable` and before `.` (?)

Often the average individual, frequently underestimate the concealed costs associated with their daily journeys:

\begin{itemize}
    \item \textbf{Time}: Commuting consumes a substantial portion of daily life that could be more productively
    allocated to work, leisure, or personal activities, thus influencing people's work and personal lives.
    \item \textbf{Money}: The expenses associated with commuting, including fuel, maintenance, parking fees, and public
    transportation fares, accumulate over time and impact individuals' financial stability, an issue that particularly
    affects the lower working class.
    \item \textbf{Climate Impact}: Commuting contributes to carbon emissions and air pollution, exacerbating
    environmental challenges, and underscores the significance of environmentally conscious transportation
    choices~\cite{alma9921355859805762}.
    \item \textbf{Health Impact}: The stress and anxiety associated with commuting, particularly in urban areas, can
    lead to various health issues, including high blood pressure, heart disease, and
    depression~\cite{commuting-and-your-health}.
\end{itemize}
