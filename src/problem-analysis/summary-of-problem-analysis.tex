\section{Summary}\label{sec:analysis-summary}

The analysis exemplifies how both the monetary cost and the concealed costs of commuting affects individuals,
as well as society.
It could be seen that price, time, environment and health all have an impact on the commuting experience.

Commuting has historically been a part of human society since ancient times, but as technological revolutions
happened with automobiles and trains urbanization began to have a stronger effect on the commuting experience.
To narrow down the scope of this paper we will choose to focus on helping commuters who need commuting for a prolonged
period of time, this could be commuting to work or for education which is something that one will need to do for
multiple years daily.
We created personas that encapsulate this target group in~\ref{subsec:persona}.
The existing solutions mainly focus on single trips some place, and are therefore focused on price and time
optimization which is not enough for repeated trips.
Through the factors that have the biggest effect on commuting, which are: price, time, environment and health.
It seems possible that a good solution to the problem would be introducing a recommendation system based on these
factors, because it would help an individual understand all the possible methods of transportation.
