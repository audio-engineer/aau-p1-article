% TODO Make sections, subsections and subsubsections much longer and remove TeXtidote ignore annotation
% textidote: ignore begin
\section{Extant solutions and their operational mechanisms}\label{sec:extant-solutions-and-their-operational-mechanisms}
% textidote: ignore end

In this section, we will discuss extant solutions and their operational mechanisms pertaining to the provided problem
statement.
We have already discussed two of the most important software for transportation one specifically focused on Denmark and
one who is focussed around the whole globe.
In this section we will be also be looking into a third software who also is focussed around the globe and that is
Apple Maps.

% textidote: ignore begin


\section{What is Apple Maps}\label{sec:what-is-apple-maps}
% textidote: ignore end

Apple Maps is mostly the same as Google Maps, it also has very detailed mapping data.
Apple Maps also provide a navigation function like Google Maps that is with turn-by-turn directions for driving, walking
and public transportation.
It also supports real-time traffic information and alternative routes to help users reach their destination more
efficiently.
Apple Maps can provide various information about shopping centers, restaurants, hotels and gas stations.
Some of that information includes user reviews, photos, contact details and hours of operation.
Google Maps does also provide this kind of information.

Google Maps used to be the default web mapping service for iOS, but they replaced Google Maps with their own version
which is of course Apple Maps now~\cite{applemaps2023}.

% textidote: ignore begin

\subsection{How can we help solve this problem?}\label{subsec:how-can-we-help-solve-this-problem?}
% textidote: ignore end

Finally, we will take a look at how we propose to solve this issue.

\subsubsection{Data collection}

We would start to gather data from ``rejseplanens'' API so that we could use all their data about the public transport
in Denmark.
We would further investigate in the environmental scene where we would look further into how much \unit{CO_{2}} emission
different car types and public transport causes per kilometer.

\subsubsection{Environmental metrics}

We would research about how much \unit{CO_{2}} emission different types of cars causes and set them to an average for
the program parameters and make it easier to calculate routes with less \unit{CO_{2}} emission.
The same goes for public transport, and we would have to research how much \unit{CO_{2}} emission trains, buses and
metros emit.

\subsubsection{Personalized preferences}

We will allow users to make different priorities such as the time of arrival, cost, environmental impact and comfort.
We would then make a filter where the user would be able to set their own priorities.

% textidote: ignore begin

\subsubsection{User interface}
% textidote: ignore end

The User Interface is supposed to be user-friendly for all people around the world, so that it can be used more by
everybody.
The GUI we are using for this project are the terminal, which is just text-based.
The reason for the choice of our GUI is limited by the projects rules in which we were recommended to do a terminal GUI
instead of another more user-friendly GUI\@.

% textidote: ignore begin

\subsubsection{Expected outcomes}
% textidote: ignore end

The expected outcome from making a service like this is that we want it cause people to think more on the environment
and make them think twice about their daily journey to work and just how much \unit{CO_{2}} emission is actually used
every time.
And making this program would also allow users to plan their journey days before, which would improve traffic flow
overall.

% textidote: ignore begin

\subsection{More about a problem solution}\label{subsec:more-about-a-problem-solution}
% textidote: ignore end
