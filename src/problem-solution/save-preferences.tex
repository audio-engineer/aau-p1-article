\section{Saving user preferences}\label{sec:saving-user-preferences}

To make the program more user-friendly, it needs to be able to store a users preferences on climate, price,
time, and so on.
This will be stored locally in a configuration file, which is called ``\url{preferences.json}''.
As the name suggests, the data is stored in a JSON format.
This is partly because the program already have a system in place to store JSON data, but also because it allows the
program to access individual keys of data such as ``price''.
Storing the preferences as JSON also allows us to remove or add different keys to the file later if needed.
However, for the sake of reducing complexity, the selected attributes to be stored are: price, time and environment.

One of dependencies of the program is therefore data for all preference categories.
The program initialize the save file with some standard default data, such that the program can continue running even if
the user does not choose to input their own preferences.
The program initializes the save file through a function called \texttt{InitializePreferenceFile}.

\begin{lstlisting}[label={lst:SavePreferences}, caption={Saving preferences to file.}]
FILE* preferences = fopen("preferences.json", "w");
fputs(kSerializedJsonWithNewline, preferences);
fclose(preferences);
free(kSerializedJsonWithNewline);

cJSON_Delete(kJsonPreferences);
\end{lstlisting}

In Figure~\ref{lst:SavePreferences} the initialization of the preference file is located.
The rest of the code inside \texttt{InitializePreferenceFile} is used to convert the given inputs into the JSON format,
which in C is somewhat difficult.

The remaining functions for the preference file are \texttt{SetUserPreference} and \texttt{GetUserPreference}.
The \texttt{SetUserPreference} function parses two inputs, a key and a desired value for that key.
This would be written as \texttt{SetUserPreference(key, value)}.
When changing the values in the JSON file through C, the program must first read the file and then parse it as cJSON[.]
Parsing JSON and generating JSON is done through the library cJSON as mentioned in Section
~\ref{subsec:libraries-and-frameworks}.
As seen in Figure~\ref{lst:SavePreferences}, the routine works by first parsing the content of the preference file and
then searching for the item to be changed.

\begin{lstlisting}[label={lst:ParseJSON}, caption={Parsing JSON.}]
const cJSON* const kJsonPreferences = cJSON_Parse(kFileBuffer);
free(kFileBuffer);

cJSON* const kJsonValue = cJSON_GetObjectItem(kJsonPreferences, key);

kJsonValue->valuedouble = value;
\end{lstlisting}

cJSON stores all JSON values as structs.
To change the value, the program accesses the struct as seen in Figure~\ref{lst:ParseJSON}.
Changing the values of a file and afterward saving to the same file location in C can get complicated if the length of
the file is changed.
Therefore, the program instead overwrites the whole save file with the updated values.

Finally, the program reads the file to access the stored data.
This is done much in the same way as in \texttt{SetUserPreference} by opening the file, parsing the JSON and then
accessing the item to be read.
Both \texttt{InitializePreferenceFile} and \texttt{SetUserPreference} are void functions.
However, \texttt{GetUserPreference} is a double function.
The reason for this is to return the value of the preference from the save file, as seen in Figure~\ref{lst:GetPref}.
\texttt{GetUserPreference} only takes one input, which is the key for the value of an attribute to be accessed.

\begin{lstlisting}[label={lst:GetPref}, caption={Reading attribute value from preference file.}]
cJSON* file_item = cJSON_GetObjectItem(json_preferences, key);

value = cJSON_GetNumberValue(file_item);

cJSON_Delete(json_preferences);

return value;
\end{lstlisting}

With these three functions it is possible for the program to store and use preferences locally on the device.
