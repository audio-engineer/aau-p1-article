% textidote: ignore begin
\chapter{Method}\label{ch:method}
% textidote: ignore end

\subsection{How can we help solve this problem?}\label{subsec:how-can-we-help-solve-this-problem?}
\subsubsection{Data collection}
We gather data from ``rejseplanens'' API such that we can use all their data about the public transport in Denmark.
We would further investigate in the environmental scene where we would look further into how much \unit{CO_{2}} emission
different car types and public transport causes per kilometer.

\subsubsection{Environmental metrics}
We would research about how much \unit{CO_{2}} emission different types of cars causes and set them to an average for
the program parameters and make it easier to calculate routes with less \unit{CO_{2}} emission.
The same goes for public transport, and we would have to research how much \unit{CO_{2}} emission trains, buses and
metros emit.

\subsubsection{Personalized preferences}
We will allow users to make different priorities such as the time of arrival, cost, environmental impact and comfort.
We would then make a filter where the user would be able to set their own priorities.

\subsubsection{User interface}
The User Interface is supposed to be user-friendly for all people around the world, so that it can be used more by
everybody.
The GUI we are using for this project are the terminal, which is just text-based.
The reason for the choice of our GUI is limited by the projects rules in which we were recommended to do a terminal GUI
instead of another more user-friendly GUI\@.

\subsubsection{Expected outcomes}
The expected outcome from making a service like this is that we want it cause people to think more on the environment
and make them think twice about their daily journey to work and just how much \unit{CO_{2}} emission is actually used
every time.
And making this program would also allow users to plan their journey days before, which would improve traffic flow
overall.
