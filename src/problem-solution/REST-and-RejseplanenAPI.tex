\section{Integrations}\label{sec:integrations}

In the following section we will take a closer look at the integrations that will help us create the software.
We will be taking a further look into what an REST-API is, and we would also be looking into how Rejseplanens API works
and how we are going to use it for this project.


\section{REST-API}\label{sec:rest-api}

To understand what a REST-API is we will look further into what a NON-REST-API is and what it does.
It will all be explained shortly what an API is.

\subsection{What is an API}\label{subsec:what-is-an-api}

Application Programming Interface (API), is a way for two or more programs to communicate with eachother.
it is a type of interface for software as it can service other pieces of software.
An API is often made up of different parts which can act as a tool for the programmer, when a programmer is using the
API they can make \textit{calls} to different parts of the API\@.
The \textit{calls} that make up the API are also known as \textit{methods, requests or endpoints}. \cite{APIwiki}

The purpose of an API is that it simplifies programming by abstracting the implementation and only exposing the objects
or actions the developer needs.
An API makes it so that a developer can press a button in a program that does something like replacing files one place
to another without the developer needing to understand the operations occurring behind the scenes.

\subsection{What is a REST-API}\label{subsec:what-is-a-rest-api}

A REST API is an API that goes by the design of REST also known as \textit{representational state transfer}
architectural style.
It is called a REST API because it uses a certain design compared to other APIs where some use different frameworks
such as SOAP where its data is strictly XML-format and where REST is more flexible being able to support a larger
framework for data transfers as REST supports different formats, including \textit{XML, HTML, plain text, JSON... and
more}~\cite{IBMrestapi}.

REST APIs communicate via HTTP requests to perform simple database functions such as \textit{creating, reading,
    updating and deleting records}~\cite{IBMrestapi}, an example could be that a REST API would want to use a GET
request to retrieve data and a POST request to create data, a PUT request to update the data and a DELETE request to
delete data.

\textbf{Key principles and characteristics of a RESTful API}
\begin{itemize}
    \item \textbf{Staleness}: Each request made from a client to a server must contain all the information needed to
    understand and process that request.
    The server should not store any information about the client's state between requests.
    This enhances scalability and simplifies server implementation.
    \item \textbf{CRUD Operations}: RESTful APIs use standard HTTP methods (GET, POST, PUT, DELETE) to perform CRUD
    operations on resources so that would be (Create, Read, Update, Delete).
    Each HTTP method corresponds to a specific action on the resource.
    \subitem \textbf{GET}: Retrieves a representation of the resource
    \subitem \textbf{POST}: Creates a new resource
    \subitem \textbf{PUT}: Updates an already existing resource
    \subitem \textbf{DELETE}: Removes an existing resource
    \item \textbf{Representation}: The resources are represented in a format, such as JSON or XML.
    The client and server has to agree on which format the data needs to be in during the communication.
    JSON and XML is often preferred because it can be read by a human and is therefore easier to navigate.
\end{itemize}


\section{Rejseplanens API}\label{sec:rejseplanens-api}
