% textidote: ignore begin
\section{Evaluating the attributes}\label{sec:evaluating-the-attributes}
% textidote: ignore end

After the attributes for each route has been calculated, the attributes are evaluated against each other, as stated in
the flowchart.
To perform these comparisons the \texttt{Evaluate} function, as seen in~\ref{lst:evaluate-prototype}, is used.

\begin{lstlisting}[caption={Function prototype for \texttt{Evaluate}}, label={lst:evaluate-prototype}, captionpos=b]
void Evaluate(TripData data_arr[], TripScore score_arr[], size_t size_arr);
\end{lstlisting}

This routine has an array of the type \texttt{TripData} as an input parameter, which is a struct type as seen in
~\ref{lst:TripData-declaration}.

\begin{lstlisting}[caption={Declaration of \texttt{TripData} struct}, label={lst:TripData-declaration}, captionpos=b]
typedef struct TripData {
  int trip_id;
  double price;
  double health;
  double time;
  double environment;
} TripData;
\end{lstlisting}

This, as well as the size of the array, allows the \texttt{Evaluate} function to read from the members served by the
\texttt{Calculating} functions mentioned earlier as well as write to the \texttt{\_score} members.
This function's primary function is to call the function \texttt{CalculateScore} function, as seen in
~\ref{lst:calculatescore-prototype}, with the different attributes of the routes as input parameters.

\begin{lstlisting}[caption={Function prototype for \texttt{CalculateScore}},label={lst:calculatescore-prototype},
    captionpos=b]
void CalculateScore(TripData trip_data[], TripScore trip_score[],
    const CalculateScoreParameters* calculate_score_parameters);
\end{lstlisting}

It does this by having the struct array mentioned before and by having an instance of the
\texttt{CalculateScoreParameters} struct, as seen in~\ref{lst:CalculateScoreParameters-declaration} as input parameters.

\begin{lstlisting}[caption={Declaration of \texttt{CalculateScoreParameters} struct},
    label={lst:CalculateScoreParameters-declaration}, captionpos=b]
typedef struct CalculateScoreParameters {
  const size_t kNumTrips;
  const size_t kReadOffset;
  const size_t kWriteOffset;
  const int kInverted;
} CalculateScoreParameters;
\end{lstlisting}

This struct is used to pass multiple input parameters without clouding the input parameters of the function.
The functionality of the \texttt{CalculateScore} function includes identifying the largest and smallest member of a
specific member type for all iterations of the struct array.
The initialization of the local variable \texttt{Score}, as seen in~\ref{lst:score_initialization}, includes the score
calculation for each instance of the struct for the relevant member passed in the \texttt{CalculateScore} function.

\begin{lstlisting}[caption={Initialization of \texttt{Score}}, label={lst:score_initialization}, captionpos=b]
double score = (*(double*)read_member - attribute_smallest) /
    (attribute_largest - attribute_smallest);
\end{lstlisting}

By subtracting the smallest member from the value of each member, as well as the largest member, the values are shifted
to a common baseline of \texttt{0}.
The score for the relevant member for each route is then given relative to the largest and smallest member, hereby
given one member the score of \texttt{1}, another the score of \texttt{0} and the rest the relative score in between.

This process is as mentioned iterated over all member types in the \texttt{TripData} struct array.

Finally, the \texttt{overall\_score} member is calculated for all the iterations in the struct array, as seen in
~\ref{lst:overall_score-calculation}.

\begin{lstlisting}[caption={Calculation of \texttt{overall\_score}}, label={lst:overall_score-calculation},
    captionpos=b]
for (size_t i = 0; i < size_arr; i++) {
  score_arr[i].trip_id = data_arr[i].trip_id;
  score_arr[i].overall_score =
      GetUserPreference("price") * score_arr[i].price_score +
      GetUserPreference("health") * score_arr[i].health_score +
      GetUserPreference("time") * score_arr[i].time_score +
      GetUserPreference("environment") * score_arr[i].environment_score;
\end{lstlisting}




