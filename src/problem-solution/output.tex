\section{Output functionality}\label{sec:output-functionality}

The last routine of the software is the~\texttt{Output} routine as mentioned in the flowchart in figure
~\ref{fig:output}.
As seen in listing~\ref{lst:output-prototype} in line 1, the~\texttt{Output} routine has the~\texttt{TripData} struct
and the~\texttt{TripScore} struct as input parameters.
This is because the overall functionality of the functions is to print the information in these structs to the terminal.
The~\texttt{Output} routine is one big while-loop iterating as long as the user wants to use the program and only
terminates when the user inputs~\texttt{Q} to quit.

\begin{lstlisting}[caption={Function prototype for~\texttt{Output}}, label={lst:output-prototype}, captionpos=b]
void Output(TripData data_arr[], TripScore score_arr[], size_t size);
\end{lstlisting}

The data to be printed in the~\texttt{Output} routine is either the data and scores for the routes that score in the top
5 or the details of a specific route.
As seen in the flowchart~\ref{fig:output}, what is being printed is reliant of the user input.
When printing the price, time and environment data and scores, the function~\texttt{PrintTripScoresAndData}
~\ref{lst:printtripscoresanddata-prototype} handles the printing to the terminal of these.

\begin{lstlisting}[caption={Function prototype for~\texttt{PrintTripScoresAndData}},
    label={lst:printtripscoresanddata-prototype}, captionpos=b]
void PrintTripScoresAndData(TripData data_arr[], TripScore score_arr[], size_t size);
\end{lstlisting}

However, when printing the route details, the function~\texttt{PrintRouteDetails}
~\ref{lst:printroutedetails-prototype} is used.
This function iterates over all legs of a given trip and outputs the route details continually until all legs are
printed.

\begin{lstlisting}[caption={Function prototype for~\texttt{PrintRouteDetails}},
    label={lst:printroutedetails-prototype}, captionpos=b]
void PrintRouteDetails(Trips* trips, long choice, TripScore score_arr[], size_t number_trips);
\end{lstlisting}

Another important part of the~\texttt{Output} routine is the sorting of the scores.
This sorting are handled by the~\texttt{SortRoutes} function.
This function sorts the entries in the~\texttt{TripScore} struct, which is where the scores for the routes are located.

\begin{lstlisting}[caption={Function prototype for~\texttt{SortRoutes}}, label={lst:sortroutes-prototype},
    captionpos=b]
void SortRoutes(TripScore* trips, char* attribute, size_t size_of_struct_array)
\end{lstlisting}
