% textidote: ignore begin
\section{Flowchart of the program}
% textidote: ignore end

A flowchart is a process that documents a program's process and becomes
interesting when breaking down the program's steps.
Basically, it serves as a diagram illustrating the programming algorithms.
This is crucial because programming processes can often appear
complex and confusing at first glance.
By depicting how the program progresses through various possibilities,
developers gain a much clearer visual overview of the end product.
Flowcharts consist of different shapes connected by arrows.
For instance, a rectangle with rounded corners is utilized as a
terminal/terminator, representing the start and stop of the main program.
Our team followed a comprehensive guide for creating these flowcharts.
In addition to the main program's overview flowchart,
we meticulously described our inputs using parallelograms, symbolizing input/output operations.
This detailed process significantly contributed to our development by providing the team with clarity.
% TODO: skriv hvad en flowchart er og hvorfor / hvordan vi bruger den; fjerne textidote ignore fra toppen

% textidote: ignore begin
%-----------------------------------------------------------------------------------------------------------------------

\subsection{Main}
\noindent
\begin{tikzpicture}[node distance=2cm]

    \node (start) [startstop] {Start};
    \node (func1) [function, right of=start, xshift=1.5cm] {Input};
    \node (func2) [function, below of=func1, yshift=-1cm] {Search};
    \node (func3) [function, left of=func2, xshift=-1.5cm] {Calculate};
    \node (func4) [function, left of=func3, xshift=-1.5cm] {Sort};
    \node (func5) [function, below of=func4, yshift=-1cm] {Output};
    \node (stop)  [startstop, right of=func5, xshift=1.5cm] {Stop};

    \draw [arrow] (start) -- (func1);
    \draw [arrow] (func1) -- (func2);
    \draw [arrow] (func2) -- (func3);
    \draw [arrow] (func3) -- (func4);
    \draw [arrow] (func4) -- (func5);
    \draw [arrow] (func5) -- (stop);

\end{tikzpicture}
%-----------------------------------------------------------------------------------------------------------------------

\subsection{Input}
\noindent
\begin{tikzpicture}[node distance=2cm]

    \node (start) [startstop] {Start};
    \node (in1)   [io, below of=start] {Read start and end destination};
    \node (in2)   [io, below of=in1] {Read time of departure / arrival};
    \node (in3)   [io, below of=in2, text width=4cm, yshift=-1cm] {Read user choice:\\ set preferences (Y)\\ no
    preferences (N)\\ open custom file (F)};
    \node (dec1)  [decision, below of=in3, yshift=-2cm] {Check choice};
    \node (pro1)  [process, left of=dec1, xshift=-2.5cm] {Set preferences according to the file};
    \node (in4)   [io, right of=in1, xshift=4cm] {Read transport selection};
    \node (dec2)  [decision, below of=in4, yshift=-1cm] {Car selected?};
    \node (in5)   [io, below of=dec2, yshift=-1cm] {Read fuel efficiency};
    \node (in6)   [io, below of=in5] {Read the priorities};
    \node (dec3)  [decision, below of=in6, yshift=-1cm] {Save?};
    \node (pro2)  [process, below of=dec3, yshift=-1cm] {Save current preferences to a file};
    \node (stop) [startstop, below of=dec1, yshift=-3cm] {Stop};

    \coordinate [right of=dec1, xshift=1cm] (ph1);
    \coordinate [left of=dec3, xshift=-1cm] (ph2);
    \coordinate [below of=ph2, yshift=-1cm] (ph3);
    \coordinate [right of=dec2, xshift=1cm] (ph4);

    \draw [arrow] (start) -- (in1);
    \draw [arrow] (in1) -- (in2);
    \draw [arrow] (in2) -- (in3);
    \draw [arrow] (in3) -- (dec1);
    \draw [arrow] (dec1) -- node[anchor=south] {F} (pro1);
    \draw [arrow] (pro1) |- (stop);
    \draw [arrow] (dec1) -- node[anchor=east] {N} (stop);
    \draw [line] (dec1) -- node[anchor=south] {Y} (ph1);
    \draw [arrow] (ph1) |- (in4);
    \draw [arrow] (in4) -- (dec2);
    \draw [arrow] (dec2) -- node[anchor=east] {yes} (in5);
    \draw [arrow] (in5) -- (in6);
    \draw [line] (dec2) -- node[anchor=south] {no} (ph4);
    \draw [arrow] (ph4) |- (in6);
    \draw [arrow] (in6) -- (dec3);
    \draw [line] (dec3) -- node[anchor=south] {no} (ph2);
    \draw [line] (ph2) -- (ph3);
    \draw [arrow] (dec3) -- node[anchor=east] {yes} (pro2);
    \draw [arrow] (pro2) -- (stop);

\end{tikzpicture}
%-----------------------------------------------------------------------------------------------------------------------

\subsection{Search}
\noindent
\begin{tikzpicture}[node distance=2cm]

    \node (start) [startstop] at (2,-1){Start};
    \node (in1) [io, above of=start] at (0,0) {Start \\ Destination};
    \node (in2) [io, above of=start] at (4,0) {Time: \\ Departure \\ Arrival};
    \node (pro1) [process, below of=start] at (2,-1) {Repeat for all APIs};
    \node (pro2) [process, below of=pro1] at (2,-3){Get routes and store routes};
    \node (in3) [io, left of=pro2] at (-1,-5){Output all routes};

    \draw [arrow] (in1) |-  (start);
    \draw [arrow] (in2) |-  (start);
    \draw [arrow] (start) -- (pro1);
    \draw [arrow] (pro1) -- (pro2);
    \draw [arrow] (pro2) -- (in3);

\end{tikzpicture}
%-----------------------------------------------------------------------------------------------------------------------

\subsection{Evaluate}
\noindent
\begin{tikzpicture}[node distance=2cm]

    \node (start) [startstop] {Start};
    \node (in1) [io, right of=start] at (4,0) {Routes};
    \node (in2) [io, above of=in1] {Preferences};
    \node (pro1) [process, below of=start] {Calculate attributes for routes};
    \node (pro2) [process, below of=in1] {Evaluate routes using: \\ Attributes \\ Preferences};
    \node (pro3) [process, below of=pro2] {Return Routes with numerical rating/evaluation};
    \node (stop) [startstop, below of=pro1] {Stop};

    \draw [arrow] (in1) -- (start);
    \draw [arrow] (in2) -| (start);
    \draw [arrow] (start) -- (pro1);
    \draw [arrow] (pro1) -- (pro2);
    \draw [arrow] (pro2) -- (pro3);
    \draw [arrow] (pro3) -- (stop);
\end{tikzpicture}
%-----------------------------------------------------------------------------------------------------------------------

\subsection{Sort}
\noindent
\begin{tikzpicture}[node distance=2cm]

    \node (start) [startstop] {Start};
    \node (pro1) [process, below of=start] {Sort routes using calculated rating of routes};
    \node (stop) [startstop, below of=pro1] {Stop};

    \draw [arrow] (start) -- (pro1);
    \draw [arrow] (pro1) -- (stop);

\end{tikzpicture}
%-----------------------------------------------------------------------------------------------------------------------

\subsection{Output}
\noindent
\begin{tikzpicture}[node distance=2cm]

    \node (start) [startstop] {Start};
    \node (out1)  [io, below of=start] {Print search results};
    \node (in1)   [io, below of=out1, text width=4cm, yshift=-0.5cm] {Read user choice:\\ view route details (1-5)\\
    sort the list (P,T,S,C)};
    \node (dec1)  [decision, below of=in1, yshift=-1.5cm] {Check int/char};
    \node (pro1)  [process, left of=dec1, xshift=-2.5cm] {Sort the list according to the input character};
    \node (func1) [function, below of=pro1, yshift=-1cm] {Sort};
    \node (out2)  [io, right of=dec1, xshift=2.5cm] {Print details for chosen route};
    \node (in2)   [io, below of=out2, text width=4cm, yshift=-0.5cm] {Read user choice:\\ go back to list (B)\\ quit
    the program (Q)};
    \node (dec2)  [decision, below of=in2, yshift=-1.5cm] {Check choice};
    \node (stop)  [startstop, left of=dec2, xshift=-2.5cm] {Stop};

    \coordinate [right of=dec2, xshift=1.5cm] (ph1);

    \draw [arrow] (start) -- (out1);
    \draw [arrow] (out1) -- (in1);
    \draw [arrow] (in1) -- (dec1);
    \draw [arrow] (dec1) -- node[anchor=south] {char} (pro1);
    \draw [arrow] (pro1) -- (func1);
    \draw [arrow] (dec1) -- node[anchor=south] {int} (out2);
    \draw [arrow] (out2) -- (in2);
    \draw [arrow] (in2) -- (dec2);
    \draw [arrow] (dec2) -- node[anchor=south] {Q} (stop);
    \draw [line] (dec2) -- node[anchor=south] {B} (ph1);
    \draw [arrow] (ph1) |- (out1);

\end{tikzpicture}
%-----------------------------------------------------------------------------------------------------------------------
% textidote: ignore end
