% textidote: ignore begin
\section{Flowchart of the program}
% textidote: ignore end

A flowchart is a diagram that describes a program's process and breaks it down step by step.
It is crucial to include, as programming processes can often appear complex and confusing at first glance.
The flowchart shows how the program progresses through various possibilities, which helps developers gain a much
clearer visual overview of the program itself.
Flowcharts consist of different shapes connected by arrows.
Generally, rectangles with rounded corners are used to signal when the program starts or stops.
Regular rectangles are used for back-end - processing and parallelograms are used for front-end - input or output.
Diamonds are used for decisions and in our case we've made a circle to represent a function.
This detailed process significantly contributed to our development by providing the team with clarity.
The flowcharts are seen in Figure~\ref{fig:main} through Figure~\ref{fig:output}.

The program starts with an input function~\ref{fig:input}, which first prompts the user for their current/start
position and for their desired destination.
We then give the user a choice of whether they want to use a preset preference file, predefined default values or engage
in providing the information by responding to prompts.
If the user selects the last mentioned approach, they're prompted with various transportation types to exclude or
include in their trip.
These can vary between cars, trains, bikes, walking and so on.
In order to accurately calculate the \unit{CO_{2}} emissions, we require the user to enter their car's fuel efficiency.
The last and most important step for the user is to decide the weighing of the attributes of their trip.
% textidote: ignore begin
They must input a score from 0 through 10 for each attribute, where 0 is not important and 10 is very important.
% textidote: ignore end

The next couple of functions take place in the back-end.
In Section~\ref{fig:search}, we make calls to Rejseplanen's API, which returns the list of routes that we need.
Afterward, we use the trip information to further calculate the attributes for each route.

Then in Section~\ref{fig:evaluate} we evaluate them according to the user's preferences and assign a score to them.

Finally, we have the output in Section~\ref{fig:output}.
From there, the trips scores are sorted.
The user can choose to sort the list based on some individual attribute score or to print the route details of a
specific trip.

% textidote: ignore begin
%-----------------------------------------------------------------------------------------------------------------------

\begin{figure}[H]
    \centering
    \vspace{0.5cm}
    \begin{tikzpicture}[node distance=2cm]

        \node (start) [startstop] {Start};
        \node (func1) [function, below of=start, yshift=-0.5cm] {Input};
        \node (func2) [function, below of=func1, yshift=-1cm] {Search};
        \node (func3) [function, below of=func2, yshift=-1cm] {Evaluate};
        \node (func4) [function, below of=func3, yshift=-1cm] {Output};
        \node (stop)  [startstop, below of=func4, yshift=-0.5cm] {Stop};

        \draw [arrow] (start) -- (func1);
        \draw [arrow] (func1) -- (func2);
        \draw [arrow] (func2) -- (func3);
        \draw [arrow] (func3) -- (func4);
        \draw [arrow] (func4) -- (stop);

    \end{tikzpicture}
    \caption{Program layout}\label{fig:main}
\end{figure}

%-----------------------------------------------------------------------------------------------------------------------

\begin{figure}[H]
    \centering
    \vspace{0.5cm}
    \begin{tikzpicture}[node distance=2cm]

        \node (start) [startstop] {Start};
        \node (in1)  [io, below of=start] {Read start and end destination};
        \node (in2)  [io, below of=in1] {Read time of departure / arrival};
        \node (in3)  [io, below of=in2, text width=4cm, yshift=-1cm] {Read user choice:\\set preferences (Y)\\no
        preferences (N)\\ open custom file (F)};
        \node (dec1) [decision, below of=in3, yshift=-2cm] {Check choice};
        \node (pro1) [process, left of=dec1, xshift=-2.5cm] {Set preferences according to the file};
        \node (in4)  [io, right of=in1, xshift=4cm] {Read transport selection};
        \node (dec2) [decision, below of=in4, yshift=-1cm] {Car selected?};
        \node (in5)  [io, below of=dec2, yshift=-1cm] {Read fuel efficiency};
        \node (in6)  [io, below of=in5] {Read the priorities};
        \node (dec3) [decision, below of=in6, yshift=-1cm] {Save?};
        \node (pro2) [process, below of=dec3, yshift=-1cm] {Save current preferences to a file};
        \node (stop) [return, below of=dec1, text width=5cm, yshift=-3cm] {Return(start-id, dest-id, time, priorities, preferences, fuel)};

        \coordinate [right of=dec1, xshift=1cm] (ph1);
        \coordinate [left of=dec3, xshift=-0.5cm] (ph2);
        \coordinate [below of=ph2, yshift=-1cm] (ph3);
        \coordinate [right of=dec2, xshift=1cm] (ph4);

        \draw [arrow] (start) -- (in1);
        \draw [arrow] (in1) -- (in2);
        \draw [arrow] (in2) -- (in3);
        \draw [arrow] (in3) -- (dec1);
        \draw [arrow] (dec1) -- node[anchor=south] {F} (pro1);
        \draw [arrow] (pro1) |- (stop);
        \draw [arrow] (dec1) -- node[anchor=east] {N} (stop);
        \draw [line]  (dec1) -- node[anchor=south] {Y} (ph1);
        \draw [arrow] (ph1) |- (in4);
        \draw [arrow] (in4) -- (dec2);
        \draw [arrow] (dec2) -- node[anchor=east] {yes} (in5);
        \draw [arrow] (in5) -- (in6);
        \draw [line]  (dec2) -- node[anchor=south] {no} (ph4);
        \draw [arrow] (ph4) |- (in6);
        \draw [arrow] (in6) -- (dec3);
        \draw [line]  (dec3) -- node[anchor=south] {no} (ph2);
        \draw [line]  (ph2) -- (ph3);
        \draw [arrow] (dec3) -- node[anchor=east] {yes} (pro2);
        \draw [arrow] (pro2) -- (stop);

    \end{tikzpicture}
    \caption{Input function}\label{fig:input}
\end{figure}

%-----------------------------------------------------------------------------------------------------------------------

\begin{figure}[H]
    \center
    \begin{multicols}{2}
        \vspace{0.5cm}
        \begin{tikzpicture}[node distance=2cm]

            \node (start) [return] {Start};
            \node (pro1) [process, below of=start] {Call Rejseplanens API with the given input};
            \node (pro2) [process, below of=pro1] {Save all the results};
            \node (stop) [return, below of=pro2] {Return};

            \draw [arrow] (start) -- (pro1);
            \draw [arrow] (pro1) -- (pro2);
            \draw [arrow] (pro2) -- (stop);

        \end{tikzpicture}
        \caption{Search function}\label{fig:search}

        \vspace{7cm}

%-----------------------------------------------------------------------------------------------------------------------
        \vspace{0.5cm}
        \begin{tikzpicture}[node distance=2cm]

            \node (start) [return] {Start};
            \node (pro1) [process, below of=start] {Calculate attributes for each route};
            \node (pro2) [process, below of=pro1] {Evaluate relative scores for routes};
            \node (pro3) [process, below of=pro2] {Add a numerical rating to each route};
            \node (pro4) [process, below of=pro3, yshift=-1.0cm] {Add a numerical overall rating to each route based on user preferences};
            \node (stop) [return, below of=pro4, yshift=-1.0cm] {Return};

            \draw [arrow] (start) -- (pro1);
            \draw [arrow] (pro1) -- (pro2);
            \draw [arrow] (pro2) -- (pro3);
            \draw [arrow] (pro3) -- (pro4);
            \draw [arrow] (pro4) -- (stop);

        \end{tikzpicture}
        \caption{Evaluate function}\label{fig:evaluate}
    \end{multicols}
\end{figure}

%-----------------------------------------------------------------------------------------------------------------------

\begin{figure}[H]
    \centering
    \vspace{0.5cm}
    \begin{tikzpicture}[node distance=2cm]

        \node (start) [return] {Start};
        \node (out1)  [io, below of=pro1] {Print data and score results};
        \node (in1)   [io, below of=out1, text width=4cm, yshift=-0.5cm] {Read user choice:\\ view route details (1-5)\\
        sort the list (P,T,E,O)\\ quit program (Q)};
        \node (dec1)  [decision, below of=in1, yshift=-1.5cm] {Check int/char};
        \node (pro1)  [process, below of=start] {Sort the list};
        \node (out2)  [io, right of=dec1, xshift=2.5cm] {Print route details for chosen route};
        \node (dec2)  [decision, below of=dec1, yshift=-2.5cm] {Check choice};
        \node (stop)  [startstop, right of=dec2, xshift=2.5cm] {Stop};

        \draw [arrow] (start) -- (pro1);
        \draw [arrow] (out1) -- (in1);
        \draw [arrow] (in1) -- (dec1);
        \draw [arrow] (dec1) -- node[anchor=east] {char} (dec2);
        \draw [arrow] (pro1) -- (out1);
        \draw [arrow] (dec1) -- node[anchor=south] {int} (out2);
        \draw [arrow] (dec2) -- node[anchor=south] {Q} (stop);
        \draw [arrow] (dec2.west) -- ++(-2.5,0) node[midway, above] {P, T, E, O} |- (pro1.west);
        \draw [arrow] (out2.north) -- ++(0,0) |- (in1.east);


    \end{tikzpicture}
    \caption{Output function}\label{fig:output}
\end{figure}

%-----------------------------------------------------------------------------------------------------------------------
% textidote: ignore end
