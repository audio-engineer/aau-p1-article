\documentclass[main.tex]{subfiles}

\begin{document}
    \section{Context}\label{sec:context}
    Commuting is one of the staples, so to speak, of modern life, and a central part of the post-modern industrialized
    society.
    As cities rapidly grew after the industrial revolution and more and more large-scale workplaces became the norm in
    cities, people started commuting to work from suburban areas to the city.
    In the early stages of this development in the beginning and middle of the 20th century, commuting was a rather
    simple task: People just had to get to work in the mornings and back home in the evenings.
    Nowadays, the complexity of commuting has increased drastically with many additional factors having to be
    considered, e.g., how big of a carbon footprint is acceptable for me?
    This question especially became relevant in recent times due to the increasingly noticeable effects of climate
    change and the scientific evidence that increased CO2 emissions greatly contribute to climate change.
    Other questions the modern-day commuter might ask themselves are: Do I need/want to commute every day, or is it ok
    if I work from home part of the week?
    What should be the balance between cost and environmental protection?
    Do I live in an area where the availability of public means of transportation is scarce, so I am more inclined to
    use a car?
    One of the simpler questions someone that considers commuting could ask themselves is merely: What commuting route
    should I choose?
    With the advent of the digital age and smartphones, the answer to this question have become commuting planner
    applications, that have become increasingly popular in recent years, helping individuals plan their daily commutes
    efficiently and minimize travel-related stress.
    These applications offer various features and functionalities, but they often have shortcomings that prevent the
    commuter from more advanced planning strategies, including the lack of easy access to cost and environmental impact
    information, e.g., how much C02 will be spent on a trip, how much a trip will cost, as well as limited user control
    over how these factors are weighed in route calculations.
    Existing solutions include the (mostly) worldwide available Google Maps, the popular Danish Rejseplanen service and
    a couple of other mobile and web applications that are not location-bound but can be used internationally, like
    \url{commute.org}.
    We will now look at some existing solutions and discuss their shortcomings to better understand the root of the
    problem.

    \subsection{Google Maps}\label{subsec:google-maps}
    The above-mentioned application is a widely used travel and commuting planner that provides detailed navigation and
    route information.
    While it offers estimated travel time and distance, it lacks a direct and user-friendly way to determine the cost of
    the trip or the CO2 footprint.
    Users cannot easily access information about how much money the trip will cost, as Google Maps does not integrate
    with financial or payment apps to calculate expenses like tolls, fuel, or public transportation fees.
    This makes sense since Google tries to target a large international audience with the application, making is
    difficult or nearly impossible to cover every localized payment solution or traveling restriction.
    Similarly, it does not provide real-time data on the CO2 emissions associated with the chosen route, making it
    challenging for users to make informed decisions about their environmental impact.
    Users have very limited control over how cost and environmental factors are weighed in route calculations, as Google
    Maps primarily focuses on travel time and distance.
    The only available selection options are preferences for means of transportation and whether the route should be
    calculated with ``fewer transfers'', ``less walking'' or ``wheelchair accessible'', where all three of the options
    seem to be tailored towards people with restricted mobility.

    \subsection{Rejseplanen}\label{subsec:rejseplanen}
    In Denmark, the most popular commuting planner application by far is Rejseplanen.
    This application, just like Google Maps, is both available as a web and mobile application.
    The app lets the traveler or commuter to enter trip starting and ending locations, which it then uses to calculate a
    route using public means of transportation or as secondary options as walking routes.
    When a route is calculated, the user is presented with a list-like interface from which the details of the trip can
    be read; what means of transportation will be used, which lines, what the departure and arrival times are, and if
    there are any special restrictions on the trip.
    Getting the price of the trip is possible too, but in a primitive manner that doesn’t consider factors such as if
    the user is a tourist or a regular resident, which is an important feature since tourists might have different
    traveling strategies than regular residents, and additionally tourists might not use the prepaid Rejsekort card,
    etc.
    There is an additional option of adding your Rejsekort ID to the app, so it can calculate the trip price based on
    your card plan, but even that feature is kept rather simple, possibly because the Danish transportation authorities
    also released the DOT app, whose primary responsibility is exactly that: calculating trip prices and buying tickets.
    The user can also customize the trip’s parameters slightly, e.g., if it’s a biking or a walking trip, what the
    maximum allowed biking or walking distance should be, whether the user is a “slow”, “medium” or “fast” walker (even
    though there is no guideline for how fast exactly each of these options is), and a few additional details, but a
    feature that definitely lacks is the calculation of the trip’s total CO2 emissions, which is an important factor in
    the modern-day world.
\end{document}
