\section{Results}\label{sec:results}

This section will present the program in its working form.
The below examples are created according to the group's personal commute and preferences.
The results show the program's functionality and user experience, with the goal of how a user would interact with the
program.

\subsection{Program results}\label{subsec:program-results}

The program is divided into five main sections.
The first section asks the user to input the start and end destinations, as well as the time of either departure or
arrival.
In the example in Listing~\ref{lst:results-sec1}, the input sets the start location to Odense, the end location to
Copenhagen and the time of departure to 6:20.

\begin{lstlisting}[label={lst:results-sec1}, caption={Basic parameters input.}, captionpos=b, language={}]
Use predefined or manually input? (p = predefined or m = manual) : m
Start location : odense
End location : copenhagen
Do you want to plan routes for Departure time? (y) Or Arrival time? (n): y
Input arrival/departure hour: 6
Input arrival/departure minutes : 20
\end{lstlisting}

The second section asks the user if they want to enter their preferences for the trip, if they want to use a custom
file or if they want to use the default preferences preset.
The preferences work as a questionnaire, where the user is asked to rate their priority for each of the preferences from
1 to 10.
The preferences are described in Section~\ref{sec:make-people-aware-of-how-much-money-time-climate-they-are-using-on-commuting}.
The questionnaire also asks the user if they want to include bikes, cars, trains or walking to the results.

In the example in Listing~\ref{lst:results-sec2}, the user has chosen to input their own preferences.
They decided to prioritize price and time, but not health.
They have also decided to exclude walking and biking from the results, but include trains.

\begin{lstlisting}[label={lst:results-sec2}, caption={Custom preferences input.}, captionpos=b, language={}]
Do you want to manually input your preferenes? (m)
Load preferences from a file (f)
Use presets? (p)
       >>> m
To input your personal preferences please write a number between 0-10!
Price preference (0-10): 7
Time preference (0-10): 8
Sustainability preference (0-10): 6
Can your route include walking? (y/n): n
Can your route include biking? (y/n): n
Can your route include trains? (y/n): y
Can your route include a car? (y/n): n
\end{lstlisting}

The third section will be hidden from the user, but as this is a prototype, the information is shown.
It includes the links that the program calls to get the data from the Rejseplanen API and saves the data to struct.
The ID for the start and end location is also shown, as well as the number of routes available.
The example in Listing~\ref{lst:results-sec3} shows the output from the above examples.

\begin{lstlisting}[label={lst:results-sec3}, caption={Backend data output.}, captionpos=b, language={}]
GETTING TRIP DATA...
Request will be made to URL: https://xmlopen.rejseplanen.dk/bin/rest.exe/location?input=odense&format=json
Request will be made to URL: https://xmlopen.rejseplanen.dk/bin/rest.exe/location?input=copenhagen&format=json
Origin location ID: 008600512
Destination location ID: 008600626
Request will be made to URL: https://xmlopen.rejseplanen.dk/bin/rest.exe/trip?originId=008600512&destId=008600626&time=06:20&format=json&useBus=0
There are 3 distinct routes available for this trip.
Request will be made to URL: http://webapp.rejseplanen.dk/bin//rest.exe/journeyDetail?ref=818994%2F273104%2F788270%2F121137%2F86%3Fdate%3D19.12.23%26station_evaId%3D8600512%26format%3Djson
Leg: 1, number of stops: 10
Total number of stops on trip: 10
\end{lstlisting}

The fourth section shows the table with results of the program.
It is an overview of all the transportation options available for the commute.
The table is sorted according to the user preferences, with the overall best score on top.
The user can see how long a trip takes, how much it costs and how sustainable it is.
The user can then sort the table according to the different preferences, to see which transportation option is best
for each preference.
The example in Listing~\ref{lst:results-sec4} again shows the output from the above examples.

\begin{lstlisting}[label={lst:results-sec4}, caption={Routes table output.}, captionpos=b, language={}]
------------------------------------------------------------
Priority 1: ID:   0 P:  64.00 T: 104.00 E: 342.00 --- Ps: 1.00 Ts: 1.00 Es: 0.95 --- Os: 0.98
Priority 2: ID:   2 P: 184.00 T: 213.00 E: 483.00 --- Ps: 0.65 Ts: 0.41 Es: 0.00 --- Os: 0.37
Priority 3: ID:   1 P: 411.00 T: 288.00 E: 334.00 --- Ps: 0.00 Ts: 0.00 Es: 1.00 --- Os: 0.29
------------------------------------------------------------

View route details (number), sort the list (Price (P),Time (T), Environment (E), Overall (O)) or terminate (Q):  e

------------------------------------------------------------
Priority 1: ID:   1 P: 411.00 T: 288.00 E: 334.00 --- Ps: 0.00 Ts: 0.00 Es: 1.00 --- Os: 0.29
Priority 2: ID:   0 P:  64.00 T: 104.00 E: 342.00 --- Ps: 1.00 Ts: 1.00 Es: 0.95 --- Os: 0.98
Priority 3: ID:   2 P: 184.00 T: 213.00 E: 483.00 --- Ps: 0.65 Ts: 0.41 Es: 0.00 --- Os: 0.37
------------------------------------------------------------
\end{lstlisting}

The fifth and final section shows the user the details of the chosen route.
The user can access the details by typing the ID of the route in the terminal.
The details screen shows the user what the name of the train is, what type it is, the exact time of departure and
arrival, as well as the track number.
% TODO if we update the program to show more information here
% It also shows the result of our calculations, such as the \unit{CO_{2}} emission, the price and the time.
The example in Listing~\ref{lst:results-sec5} yet again shows the output from the above examples.

\begin{lstlisting}[label={lst:results-sec5}, caption={Route details output.}, captionpos=b, language={}]
View route details (number), sort the list (Price (P),Time (T), Environment (E), Overall (O)) or terminate (Q):  1
------------------------------------------------------------
Number of legs in the chosen trip: 1
------------------------------------------------------------
Leg 1:
Transportation name:       IC 838
Transportation type:       IC

Origin station name:       Odense St.
Origin station time:       06:26
Origin station date:       19.12.23
Origin station Track:      3

Destination station name:  Kobenhavn H
Destination station time:  07:55
Destination station date:  19.12.23
Destination station Track: 4
------------------------------------------------------------
\end{lstlisting}

% TODO if we update the program, update the output here :)

\subsection{Persona results}\label{subsec:persona-results}

The following section will show the results of the program with the preferences tuned to the different personas created
in Section~\ref{subsec:persona}.

For the first persona, the issue is that he got a new job in a different city.
Let's assume that he lives in Roskilde, and that he has to commute to Copenhagen.
Time is his upmost priority, because he wants to live his life to the fullest.
An example of the program output fit for his preferences can be seen in Listing~\ref{lst:results-persona1}.
He can then make the choice of whether the commute will take too much of his free time.

\begin{lstlisting}[label={lst:results-persona1}, caption={Output for Asger Johansen.}, captionpos=b, language={}]
Priority 1: ID:   1 P: 117.00 T: 187.00 E: 487.00 --- Ps: 1.00 Ts: 1.00 Es: 0.00 --- Os: 0.81
Priority 2: ID:   0 P: 152.00 T: 188.00 E: 452.00 --- Ps: 0.19 Ts: 1.00 Es: 0.11 --- Os: 0.68
Priority 3: ID:   2 P: 160.00 T: 391.00 E: 171.00 --- Ps: 0.00 Ts: 0.00 Es: 1.00 --- Os: 0.19
\end{lstlisting}

For the second persona, the issue is that she wants to be more sustainable.
Let's assume that she lives in Taastrup, and that she has to commute to Copenhagen.
There are a number of different transportation options available for her, such as the train, the S-train, the bus, and
she is also considering an electric car.
An example of the program output fit for her preferences can be seen in Listing~\ref{lst:results-persona2}.
The best option for her is to buy an electric car, but the S-train is also a good option.

\begin{lstlisting}[label={lst:results-persona2}, caption={Output for Josefine Madsen.}, captionpos=b, language={}]
Priority 1: ID:   1 P: 185.00 T: 127.00 E:  85.00 --- Ps: 1.00 Ts: 0.46 Es: 1.00 --- Os: 0.90
Priority 2: ID:   2 P: 425.00 T:   8.00 E: 357.00 --- Ps: 0.00 Ts: 1.00 Es: 0.28 --- Os: 0.36
Priority 3: ID:   0 P: 417.00 T: 227.00 E: 463.00 --- Ps: 0.03 Ts: 0.00 Es: 0.00 --- Os: 0.01
\end{lstlisting}

For the third persona, he wonders if taking the train would be more sustainable than taking the car.
Let's assume that he lives in Odense, and that he has to commute to Nyborg.
The distance is rather small, at only 30 km.
He wants to live a carefree life, so he is not concerned with his priorities.
Therefore, let's assume that they're split evenly.
An example of the program output fit for his preferences can be seen in Listing~\ref{lst:results-persona3}.
It turns out that taking the train would not only be more sustainable, but also cheaper and faster.
It is worth nothing however that the time does not include travel to and from the train stations.

\begin{lstlisting}[label={lst:results-persona3}, caption={Output for Martin Jensen.}, captionpos=b, language={}]
Priority 1: ID:   2 P: 148.00 T: 108.00 E:  19.00 --- Ps: 1.00 Ts: 0.85 Es: 1.00 --- Os: 0.97
Priority 2: ID:   1 P: 429.00 T: 358.00 E:  32.00 --- Ps: 0.15 Ts: 0.00 Es: 0.97 --- Os: 0.64
Priority 3: ID:   0 P: 480.00 T:  64.00 E: 453.00 --- Ps: 0.00 Ts: 1.00 Es: 0.00 --- Os: 0.19
\end{lstlisting}

% TODO if we update the program, update the output here too
