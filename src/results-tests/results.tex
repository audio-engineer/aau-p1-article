\section{Results}\label{sec:results}

This section will present the program in its working form.
The below examples are created according to the group's personal commute and preferences.
The results show the program's functionality and user experience, with the goal of how a user would interact with the
program.

The program is divided into five main sections.
The first section asks the user to input the start and end destinations, as well as the time of either departure or
arrival.
In the example in Listing~\ref{lst:results-sec1}, the input sets the start location to Odense, the end location to
Copenhagen and the time of departure to 6:20.

\begin{lstlisting}[label={lst:results-sec1}, caption={Basic parameters input.}, captionpos=b, language={}]
Use predefined or manually input? (p = predefined or m = manual) : m
Start location : odense
End location : copenhagen
Do you want to plan routes for Departure time? (y) Or Arrival time? (n): y
Input arrival/departure hour: 6
Input arrival/departure minutes : 20
\end{lstlisting}

The second section asks the user if they want to enter their preferences for the trip, if they want to use a cusom
file or if they want to use the default preferences preset.
The preferences work as a questionnaire, where the user is asked to rate their priority for each of the preferences from
1 to 10.
The preferences are described in Section~\ref{sec:make-people-aware-of-how-much-money-time-climate-they-are-using-on-commuting}.
The questionnaire also asks the user if they want to include bikes, cars, trains or walking to the results.

In the example in Listing~\ref{lst:results-sec2}, the user has chosen to input their own preferences.
The user has chosen to prioritize price and time, but not health.
The user has also chosen to exclude walking and biking from the results, but include trains.

\begin{lstlisting}[label={lst:results-sec2}, caption={Custom preferences input.}, captionpos=b, language={}]
Do you want to manually input your preferenes? (m)
Load preferences from a file (f)
Use presets? (p)
       >>> m
To input your personal preferences please write a number between 0-10!
Price preference (0-10): 7
Time preference (0-10): 8
Sustainability preference (0-10): 6
Can your route include walking? (y/n): n
Can your route include biking? (y/n): n
Can your route include trains? (y/n): y
Can your route include a car? (y/n): n
\end{lstlisting}

The third section will be hidden from the user, but as this is a prototype, the information is shown.
It includes the links that the program calls to get the data from the Rejseplanen API and saves the data to struct.
The ID for the start and end location is also shown, as well as the number of routes available.
The example in Listing~\ref{lst:results-sec3} shows the output from the above examples.

\begin{lstlisting}[label={lst:results-sec3}, caption={Backend data output.}, captionpos=b, language={}]
GETTING TRIP DATA...
Request will be made to URL: https://xmlopen.rejseplanen.dk/bin/rest.exe/location?input=odense&format=json
Request will be made to URL: https://xmlopen.rejseplanen.dk/bin/rest.exe/location?input=copenhagen&format=json
Origin location ID: 008600512
Destination location ID: 008600626
Request will be made to URL: https://xmlopen.rejseplanen.dk/bin/rest.exe/trip?originId=008600512&destId=008600626&time=06:20&format=json&useBus=0
There are 3 distinct routes available for this trip.
Request will be made to URL: http://webapp.rejseplanen.dk/bin//rest.exe/journeyDetail?ref=818994%2F273104%2F788270%2F121137%2F86%3Fdate%3D19.12.23%26station_evaId%3D8600512%26format%3Djson
Leg: 1, number of stops: 10
Total number of stops on trip: 10
\end{lstlisting}

The fourth section shows the table with results of the program.
It is an overview of all the transportation options available for the commute.
The table is sorted according to the user preferences, with the overall best score on top.
The user can see how long a trip takes, how much it costs and how sustainable it is.
The user can then sort the table according to the different preferences, to see which transportation option is best
for each preference.
The example in Listing~\ref{lst:results-sec4} again shows the output from the above examples.

\begin{lstlisting}[label={lst:results-sec4}, caption={Routes table output.}, captionpos=b, language={}]
------------------------------------------------------------
Priority 1: ID:   0 P:  64.00 T: 104.00 E: 342.00 --- Ps: 1.00 Ts: 1.00 Es: 0.95 --- Os: 0.98
Priority 2: ID:   2 P: 184.00 T: 213.00 E: 483.00 --- Ps: 0.65 Ts: 0.41 Es: 0.00 --- Os: 0.37
Priority 3: ID:   1 P: 411.00 T: 288.00 E: 334.00 --- Ps: 0.00 Ts: 0.00 Es: 1.00 --- Os: 0.29
------------------------------------------------------------

View route details (number), sort the list (Price (P),Time (T), Environment (E), Overall (O)) or terminate (Q):  e

------------------------------------------------------------
Priority 1: ID:   1 P: 411.00 T: 288.00 E: 334.00 --- Ps: 0.00 Ts: 0.00 Es: 1.00 --- Os: 0.29
Priority 2: ID:   0 P:  64.00 T: 104.00 E: 342.00 --- Ps: 1.00 Ts: 1.00 Es: 0.95 --- Os: 0.98
Priority 3: ID:   2 P: 184.00 T: 213.00 E: 483.00 --- Ps: 0.65 Ts: 0.41 Es: 0.00 --- Os: 0.37
------------------------------------------------------------
\end{lstlisting}

The fifth and final section shows the user the details of the chosen route.
The user can access the details by typing the ID of the route in the terminal.
The details screen shows the user what the name of the train is, what type it is, the exact time of departure and
arrival, as well as the track number.
% TODO if we update the program to show more information here
% It also shows the result of our calculations, such as the \unit{CO_{2}} emission, the price and the time.
The example in Listing~\ref{lst:results-sec5} yet again shows the output from the above examples.

\begin{lstlisting}[label={lst:results-sec5}, caption={Route details output.}, captionpos=b, language={}]
View route details (number), sort the list (Price (P),Time (T), Environment (E), Overall (O)) or terminate (Q):  1
------------------------------------------------------------
Number of legs in the chosen trip: 1
------------------------------------------------------------
Leg 1:
Transportation name:       IC 838
Transportation type:       IC

Origin station name:       Odense St.
Origin station time:       06:26
Origin station date:       19.12.23
Origin station Track:      3

Destination station name:  Kobenhavn H
Destination station time:  07:55
Destination station date:  19.12.23
Destination station Track: 4
------------------------------------------------------------
\end{lstlisting}

% TODO if we update the program, update the output here :)
