\section{Results}\label{sec:results}

This section will present the program in its working form.
The below examples are created according to the group's personal commute and preferences.
The results show the program's functionality and user experience, with the goal of how a user would interact with the
program.

\subsection{Program results}\label{subsec:program-results}

The program is divided into five main sections.
The first two sections are dedicated to the user input.
The program starts by asking the user if they want to use a preferences file or if they want to input the parameters
manually.
If the user selects manual input, it asks them to input the start and end destinations, as well as the time of either
departure or arrival.
In the example in Listing~\ref{lst:results-sec1}, the input sets the start location to Odense, the end location to
Copenhagen and the time of departure to 6:20.

\begin{lstlisting}[label={lst:results-sec1}, caption={Basic parameters input.}, captionpos=b, language={}]
Would you like to initialize using the values saved in your preferences file, or manually input them?
       Enter 'f' for file values or 'm' for manual input: m
What are the origin and destination for your trip?
       Enter origin location: odense
       Enter destination location: copenhagen
Do you want the trip to be planned according to a departure time, or arrival time?
       Enter 'd' for departure or 'a' for arrival: d
Specify your preferred time constraints.
       Enter departure hour (0 - 23): 6
       Enter departure minute (0 - 59): 20
\end{lstlisting}

The second section is dedicated to the user preferences.
The preferences work as a questionnaire, where the user is asked to rate their priority for each of the preferences from
1 to 10.
The preferences are described in Section~\ref{sec:make-people-aware-of-how-much-money-time-climate-they-are-using-on-commuting}.
The questionnaire also asks the user if they want to include bikes, cars, trains or walking to the results.
In the example in Listing~\ref{lst:results-sec2}, the user has chosen to input their own preferences.
They decided to prioritize price and time, but not health.
They have also decided to exclude walking and biking from the results, but include trains.
At the end, the user can decide to save their preferences to a file, so they can be used later.

\begin{lstlisting}[label={lst:results-sec2}, caption={Custom preferences input.}, captionpos=b, language={}]
Now assign weights (from 1 to 10) to price, time, and sustainability in your choice of transport.
       Price (0 - 10): 7
       Time (0 - 10): 8
       Sustainability (0 - 10): 6
What modes of transport can your route include?
       Walking? (y/n): n
       Biking? (y/n): n
       Trains? (y/n): y
       Car? (y/n): n
Would you like to save these preferences to the preferences file?
       Save to file? (y/n): y
\end{lstlisting}

The third section will be hidden from the user, but as this is a prototype, the information is shown.
It includes the links that the program calls to get the data from the Rejseplanen API and saves the data to struct.
The ID for the start and end location is also shown, as well as the number of routes available.
The example in Listing~\ref{lst:results-sec3} shows the output from the above examples.

\begin{lstlisting}[label={lst:results-sec3}, caption={Backend data output.}, captionpos=b, language={}]
Request will be made to URL: https://xmlopen.rejseplanen.dk/bin/rest.exe/location?input=odense&format=json
Request will be made to URL: https://xmlopen.rejseplanen.dk/bin/rest.exe/location?input=copenhagen&format=json
Origin location ID: 008600512
Destination location ID: 008600626
Request will be made to URL: https://xmlopen.rejseplanen.dk/bin/rest.exe/trip?originId=008600512&destId=008600626&format=json&time=8:20&searchForArrival=1&useBus=0
There are 3 distinct routes available for this trip.
Request will be made to URL: http://webapp.rejseplanen.dk/bin//rest.exe/journeyDetail?ref=319389%2F107373%2F973642%2F380358%2F86%3Fdate%3D20.12.23%26station_evaId%3D8600512%26format%3Djson
Leg: 1, number of stops: 16
Total number of stops on trip: 16
\end{lstlisting}

The fourth section shows the table with results of the program.
It is an overview of all the transportation options available for the commute.
The table is sorted according to the user preferences, with the overall best score on top.
The user can see how long a trip takes, how much it costs and how sustainable it is.
The user can then sort the table according to the different preferences, to see which transportation option is best
for each preference.
The example in Listing~\ref{lst:results-sec4} again shows the output from the above examples.

\begin{lstlisting}[label={lst:results-sec4}, caption={Routes table output.}, captionpos=b, language={}]
------------------------------------------------------------
Priority 1: ID:   0 P: 171.00 T: 363.00 E:  39.00 --- Ps: 1.00 Ts: 0.27 Es: 1.00 --- Os: 0.72
Priority 2: ID:   1 P: 432.00 T:  27.00 E:  57.00 --- Ps: 0.00 Ts: 1.00 Es: 0.96 --- Os: 0.65
Priority 3: ID:   2 P: 401.00 T: 489.00 E: 460.00 --- Ps: 0.12 Ts: 0.00 Es: 0.00 --- Os: 0.04
------------------------------------------------------------

       View route details (number), sort the list (Price (P),Time (T), Environment (E), Overall (O)) or terminate (Q):  e

------------------------------------------------------------
Priority 1: ID:   0 P: 171.00 T: 363.00 E:  39.00 --- Ps: 1.00 Ts: 0.27 Es: 1.00 --- Os: 0.72
Priority 2: ID:   1 P: 432.00 T:  27.00 E:  57.00 --- Ps: 0.00 Ts: 1.00 Es: 0.96 --- Os: 0.65
Priority 3: ID:   2 P: 401.00 T: 489.00 E: 460.00 --- Ps: 0.12 Ts: 0.00 Es: 0.00 --- Os: 0.04
------------------------------------------------------------
\end{lstlisting}

The fifth and final section shows the user the details of the chosen route.
The user can access the details by typing the ID of the route in the terminal.
The details screen shows the user what the name of the train is, what type it is, the exact time of departure and
arrival, as well as the track number.
The example in Listing~\ref{lst:results-sec5} yet again shows the output from the above examples.

\begin{lstlisting}[label={lst:results-sec5}, caption={Route details output.}, captionpos=b, language={}]
       View route details (number), sort the list (Price (P),Time (T), Environment (E), Overall (O)) or terminate (Q):  1

------------------------------------------------------------
Number of legs in the chosen trip: 1
------------------------------------------------------------
Leg 1:
Transportation name:       ICL 608
Transportation type:       LYN

Origin station name:       Odense St.
Origin station time:       06:10
Origin station date:       20.12.23
Origin station Track:

Destination station name:  Kobenhavn H
Destination station time:  07:35
Destination station date:  20.12.23
Destination station Track:
------------------------------------------------------------

       View route details (number), sort the list (Price (P),Time (T), Environment (E), Overall (O)) or terminate (Q):  q
\end{lstlisting}

\subsection{Persona results}\label{subsec:persona-results}

The following section will show the results of the program with the preferences tuned to the different personas created
in Section~\ref{subsec:persona}.

For the first persona, the issue is that he got a new job in a different city.
Let's assume that he lives in Roskilde, and that he has to commute to Copenhagen.
Time is his upmost priority, because he wants to live his life to the fullest.
An example of the program output fit for his preferences can be seen in Listing~\ref{lst:results-persona1}.
He can then make the choice of whether the commute will take too much of his free time.

\begin{lstlisting}[label={lst:results-persona1}, caption={Output for Asger Johansen.}, captionpos=b, language={}]
Priority 1: ID:   1 P: 117.00 T: 187.00 E: 487.00 --- Ps: 1.00 Ts: 1.00 Es: 0.00 --- Os: 0.81
Priority 2: ID:   0 P: 152.00 T: 188.00 E: 452.00 --- Ps: 0.19 Ts: 1.00 Es: 0.11 --- Os: 0.68
Priority 3: ID:   2 P: 160.00 T: 391.00 E: 171.00 --- Ps: 0.00 Ts: 0.00 Es: 1.00 --- Os: 0.19
\end{lstlisting}

For the second persona, the issue is that she wants to be more sustainable.
Let's assume that she lives in Taastrup, and that she has to commute to Copenhagen.
There are a number of different transportation options available for her, such as the train, the S-train, the bus, and
she is also considering an electric car.
An example of the program output fit for her preferences can be seen in Listing~\ref{lst:results-persona2}.
The best option for her is to buy an electric car, but the S-train is also a good option.

\begin{lstlisting}[label={lst:results-persona2}, caption={Output for Josefine Madsen.}, captionpos=b, language={}]
Priority 1: ID:   1 P: 185.00 T: 127.00 E:  85.00 --- Ps: 1.00 Ts: 0.46 Es: 1.00 --- Os: 0.90
Priority 2: ID:   2 P: 425.00 T:   8.00 E: 357.00 --- Ps: 0.00 Ts: 1.00 Es: 0.28 --- Os: 0.36
Priority 3: ID:   0 P: 417.00 T: 227.00 E: 463.00 --- Ps: 0.03 Ts: 0.00 Es: 0.00 --- Os: 0.01
\end{lstlisting}

For the third persona, he wonders if taking the train would be more sustainable than taking the car.
Let's assume that he lives in Odense, and that he has to commute to Nyborg.
The distance is rather small, at only 30 km.
He wants to live a carefree life, so he is not concerned with his priorities.
Therefore, let's assume that they're split evenly.
An example of the program output fit for his preferences can be seen in Listing~\ref{lst:results-persona3}.
It turns out that taking the train would not only be more sustainable, but also cheaper and faster.
It is worth nothing however that the time does not include travel to and from the train stations.

\begin{lstlisting}[label={lst:results-persona3}, caption={Output for Martin Jensen.}, captionpos=b, language={}]
Priority 1: ID:   2 P: 148.00 T: 108.00 E:  19.00 --- Ps: 1.00 Ts: 0.85 Es: 1.00 --- Os: 0.97
Priority 2: ID:   1 P: 429.00 T: 358.00 E:  32.00 --- Ps: 0.15 Ts: 0.00 Es: 0.97 --- Os: 0.64
Priority 3: ID:   0 P: 480.00 T:  64.00 E: 453.00 --- Ps: 0.00 Ts: 1.00 Es: 0.00 --- Os: 0.19
\end{lstlisting}
