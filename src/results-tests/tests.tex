\section{Tests}\label{sec:tests}

After the development, the group used a qualitative evaluation method to determine the program's usability from the
user's perspective.
The method is called ``Thinking Aloud'', and the goal is to evaluate the program's UX by having the user say all their
thoughts out loud as they're using the program.
Due to the time constraints, the group only had time to test the program with one user.
The user ran two tests, first one was without any guidance, and the second one was with the group's guidance.
The candidate is Lucas Ilmari Puck Pedersen.
He lives in Odense, he is 19 years old, and he is currently holding a gap year.
All the knowledge he had for the first test is that the program is about commuting.

At the first glance of the program, he liked the look of the terminal interface.
It took him a while to figure out what the text said, he thought it was saying ``COMPUTING'', while it was actually
saying ``COMMUTING''.
He was confused about what ``file'' meant, so he started by pressing ``m'' for ``manual''.
As he doesn't commute a lot, he decided to test the program with a trip from Copenhagen Airport to Aalborg Airport.
He was then met with the question of what he prioritized the most.
That was pleasantly surprising to him and thought it was a nice feature.
He wanted a cheap trip, so he entered a low number.
In reality however, a higher number resulted in a cheaper price.
That could be something that the program could communicate better.
He also wanted a fast trip, so he entered a high number.
And for sustainability, he entered an average number.
Listing~\ref{lst:tests-1} shows the result of the first test.
Lucas decided to choose the third trip out of the four, as it was the cheapest.
He then looked at the details, took note of the time, and then he was satisfied with the result.

\begin{lstlisting}[label={lst:tests-1}, caption={Copenhagen Airport to Aalborg Airport.}, captionpos=b, language={}]
Priority 1: ID:   0 P: 113.00 T:  54.00 E:   6.00 --- Ps: 0.84 Ts: 1.00 Es: 1.00 --- Os: 0.96
Priority 2: ID:   2 P:  93.00 T: 476.00 E: 164.00 --- Ps: 0.90 Ts: 0.00 Es: 0.67 --- Os: 0.42
Priority 3: ID:   3 P:  61.00 T: 425.00 E: 487.00 --- Ps: 1.00 Ts: 0.12 Es: 0.00 --- Os: 0.34
Priority 4: ID:   1 P: 394.00 T: 375.00 E:  65.00 --- Ps: 0.00 Ts: 0.24 Es: 0.88 --- Os: 0.32
\end{lstlisting}

During the second test, Lucas had a little more experience.
This time he attempted to recreate a real world scenario.
He has a friend that travels to university every day from Odense to Copenhagen, so he entered those stations as input.
University starts at 8:00, so he would want to be at the station at 7:30.
So he entered 7:30 as the time of arrival.
For the preferences, he prioritized price and time.
The output can be seen in Listing~\ref{lst:tests-2}.
He chose the second trip, as it was the fastest.

\begin{lstlisting}[label={lst:tests-2}, caption={Odense St. To Copenhagen St.}, captionpos=b, language={}]
Priority 1: ID:   1 P:  86.00 T: 395.00 E: 345.00 --- Ps: 1.00 Ts: 0.00 Es: 0.96 --- Os: 0.61
Priority 2: ID:   2 P: 309.00 T: 198.00 E: 372.00 --- Ps: 0.21 Ts: 1.00 Es: 0.00 --- Os: 0.51
Priority 3: ID:   0 P: 368.00 T: 325.00 E: 344.00 --- Ps: 0.00 Ts: 0.36 Es: 1.00 --- Os: 0.19
\end{lstlisting}
