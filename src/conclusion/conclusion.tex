% textidote: ignore begin
\chapter{Conclusion}\label{ch:conclusion}
% textidote: ignore end
The problems surrounding commuting are both large in scope and complicated to solve.
A large part of commuting is the many different ways you can go about traveling.
As an example take Google Maps, where you can already search for both car, public transportation, biking and walking.
You can always see the fastest way to get to some place with all the stated methods, but if this was how one was to
commute daily for multiple years would it not be interesting and important to now in more detail what this would cost?
As written about in~\ref{sec:shortcomings-of-commuting}, these costs are not only monetary, but have other
aspects that should be prioritized when deciding the means of transportation, which current solutions do not help
meaningfully with.

Commuting is also important because of the shear amount of people who do it daily, because of urbanization most people
live in proximity to others, while still needing to commute a meaningful distance to get to work.
This puts extra strain on highways and roads creating the congestion bigger cities and highways are known for.
Through research, it became clear that the most important factors for choosing travel are: price, time, health,
including both mental and physical health, and a newer entry which is the environmental impact.
Price is the factor most people consciously think about when choosing how to commute, as some people cannot afford the
more expensive types of transportation, an example of this would be car travel.
Time is also a factor, but not many people plan around it as long as the distance is within a threshold set by the
individual.
Both health and environment are harder to find information about and therefore more difficult to take into consideration
when deciding how to commute, but if given the choice most people could consider changing commute if they could easily
get information on what difference it would make both for themselves and the environment if they chose a different route
or a different transportation method.

The problem we want to solve is a problem related to knowledge.
With new and modern online maps and trip planners, getting a route and a price is not the hard part, but to get a deeper
understanding of the different routes and types of transportation requires some amount of research.
Knowledge is power or so the saying goes, this is also what could be key in helping humans as a society in choosing
the best possible ways to get from point A to point B\@.
Given the choice of the factors: price, time, health and environment, it becomes clear that if a program could be made,
that takes these different factors/attributes into account when searching, or sorting some amount of given routes, it
would be a big step in helping people get an overview of their possible choices when it comes to commuting.
Another added bonus of giving the user knowledge about their commute is that they can help move society forward, by
fighting for better alternatives and more investment into this by governments.
