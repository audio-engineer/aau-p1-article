\chapter{Conclusion}\label{ch:conclusion}

The problems surrounding commuting are both large in scope and complicated to solve.
A large part of commuting is the many ways you can go about traveling.
An example is Google Maps, where one can already search for both car, public transportation, biking and walking.
It provides the fastest way to get to some place with all the stated methods.
However, if this was how one was to commute daily for multiple years, it would be more interesting and important to know
in more detail what this would cost for the user and for the environment.
As written about in~\ref{sec:shortcomings-of-commuting}, these costs are not only monetary, but also contain other
aspects that should be prioritized when deciding the means of transportation, which current solutions do not help
meaningfully enough with.

Commuting is also significant because of the shear amount of people who do it daily.
Because of urbanization most people live in proximity to others, while still needing to commute a meaningful distance to
get to work.
This puts extra strain on highways and roads creating the congestion that bigger cities and highways are known for.
It is clear that the most essential factors for most users for deciding on a travelling method are: price, time,
environment and health, including both mental and physical health.
Price is the factor most people consciously consider when choosing how to commute, as some people cannot afford the
more expensive types of transportation.
Both health and environment are hard to quantify and therefore more difficult to take into consideration when deciding
how to commute.
However, if given the choice, more people could consider changing commute, if they could easily receive information on
what difference it would make for both themselves and the environment, if they decide on a different route or a
different transportation method.

The problem we wanted to solve is a problem related to knowledge and practicality.
With new and modern online maps and trip planners, getting a route and a price is not difficult.
However, to get a deeper understanding of the attributes mentioned earlier out program is the solution.
Knowledge is power, or so the saying goes.
This is also what could be key in helping humans as a society in selecting the best possible ways to get from point A to
point B\@.
Given the choice of the factors: price, time, health and environment, it becomes clear that if a more complete version
of our program was made, that takes even more different factors/attributes into account when searching for routes, it
would be a big step in helping people get an overview of their possible choices when it comes
to commuting.
