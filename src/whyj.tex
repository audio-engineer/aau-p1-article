\section{Relevance of the problem}
Commuting is a part of our everyday life and especially of people in employment.
A study from Denmark’s statistic clearly shows that out of around 3 million people
only 185 thousand people doesn’t commute to work.
This means 83,65\% commute to work in Denmark.
As we mentioned earlier people are uninformed about how much emissions
they are using while choosing their desired commuting option.
The desired preference of the commuting can change the way people commutes.
A study made in 2021 on commuting showed which transport option users would
choose depending on what their preferences were.
Here we can clearly see that public transport is preferred when the priority is
saving time, cost or being environmentally friendly while walking
or cycling were more preferred when health was the priority.
This is interesting because if people had the opportunity to choose either economy,
time or sustainability as a preference when commuting they could use alternative transport options.
Furthermore, this could possibly promote sustainable commuting. This may be important as sustainable living
is becoming more trending than ever.
This is a direct result of people becoming more aware of how their choices
can have an effect on the environment they live in.
Research done by Capgemini research shows a sturdy connection between sustainability and important business benefits.
This means that that customer loyalty and revenue increases by having a focus on being sustainable.
The enlarged focus on sustainability is because the future shoppers
are the new youth, and they prioritize sustainability more than the millennials.
This shows that it could be environmentally beneficial if the people had a saying in how they prefer to commute.
Commuting can also influence mental health and well-being as it can cause an increase in stress level.
This is apparent in some research. One suggests that one of the factors
is that the commuters have a difficulty in appreciating their way of commuting.
The author furthermore explains that the governments must prioritize what the commuters prefer.
Beside these workers can have an incline in stress levels as they have deadlines and specific meeting times.
By this we can derive that having the opportunity to travel to
your destination fastest as possible can be beneficial for the health.
Another research made a study on 208 people who were rail commuters In New York and made an interesting discovery.
It showed that the longer time that they commuted the more their cortisol level increased.
This means that time duration of commuting is an important factor.
Therefore, it is important that people commuting can have an influence on time spent commuting.
Another important aspect is how commuting effect the economy of people.
Research shows that the average American citizen spends \$8,466 on co commute expenses.
There is different cost involved depending on which transport options you use.
For example, 76\% of commuters use their personal vehicle.
This have some extra expenses besides buying a car.
The average driver spends \$1,771 on full car insurance and furthermore
they spend money on car maintenance and occasionally after 5000 miles they need to get their vehicles' oil changed.
This could be a challenge as research shows that 1 out of 3 drivers can’t afford sudden unexpected car repairs.
The study suggests that using public transport, bike or scooter to work will cost less.
This change could be difficult as commuting in a personal vehicle could be much more time efficient.
A study showed that the working poor who commuted by public transport spend 10\%
of their income on commuting compared to 21\% when commuting in their car.
The working poor prefer saving money because this means they can spend a larger portion of their income
on other expenses like food, healthcare, household supplies and saving for the future.
From this, it is apparent that especially the poorer people could have an advantage by having the opportunity
to commute by their preferences.
This could improve their economy as the working poor tend to use the least expensive options of
commuting, like carpooling, vanpooling, biking, walking and public transport.
Comparing this to people with higher income they are more inclined to spend more on commuting.
