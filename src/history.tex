People have been commuting since the beginning of humanity.
From fetching berries in the forest to fighting in wars.
In more recent human history commuting encapsulates something more complex.
This trend reversed as cars and other more technologically advanced vehicles became accessible.
As most worker were from a lower socioeconomic class with limited resources to spend on transport and luxury, it only made sense for workers to move closer to the factory they worked at.
As the Netherlands, as well as most other European countries, experienced the same technological breakthroughs in the same relative time frames in history, the country is in broad sense comparable to Denmark.
Commuting in the rural areas of the country is heavily influenced by cars, as the population density is dramatically lower than in the big cities.This trend reversed as cars and other more technologically advanced vehicles became accessible.
As most worker were from a lower socioeconomic class with limited resources to spend on transport and luxury, it only made sense for workers to move closer to the factory they worked at.
As the Netherlands, as well as most other European countries, experienced the same technological breakthroughs in the same relative time frames in history, the country is in broad sense comparable to Denmark.
Commuting in the rural areas of the country is heavily influenced by cars, as the population density is dramatically lower than in the big cities.
