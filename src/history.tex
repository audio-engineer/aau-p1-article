\chapter{The history of commuting}
People have been commuting since the beginning of humanity. 
From fetching berries in the forest to fighting in wars. 
In more recent human history commuting encapsulates something more complex.
Before the information age in the 20\textsuperscript{th} century and before cars and busses were mainstream transport vehicles, most people traveled on foot or on animal to and from work.
This trend reversed as cars and other more technologically advanced vehicles became accessible.
In the early 19 hundreds Netherlands, as described in the book No bicycle, No bus, No job, people would flock to the cities to avoid unemployment as the industrial evolution was in full force.
As most worker were from a lower socioeconomic class with limited resources to spend on transport and luxury, it only made sense for workers to move closer to the factory they worked at.
As the Netherlands, as well as most other European countries, experienced the same technological breakthroughs in the same relative time frames in history, the country is in broad sense comparable to Denmark.
As cars and busses populated the streets in the 19 hundreds, commuting slowly became the complex issue as we know it today.
Commuting in the rural areas of the country is heavily influenced by cars, as the population density is dramatically lower than in the big cities.
\cite{Befolkningstal}
