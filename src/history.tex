\chapter{The history of commuting} 
The problem of choosing public commuting, private commuting or moving closer to one’s place of work is a more modern privileged western first world problem. 
People have been commuting since the beginning of humanity. 
From fetching berries in the forest to traveling to fighting in wars. 
In more recent human history commuting encapsulates something more complex.  
Before the information age in the 20th century and before cars and buses and such were mainstream transport vehicles, most people traveled on foot or were accompanied by some animal suited for travel to and from work. 
This trend reversed as cars and other more technologically advanced vehicles became accessible. 
In the early 19 hundred Netherlands, as described in the book No bicycle, No bus, No job, people would flock to the cities to avoid unemployment as the industrial evolution was in full force. 
As most workers were from a lower socioeconomic class with limited resources to spend on transport and luxury, it only made sense for workers to move closer to the factory they worked at to save time and money. 
