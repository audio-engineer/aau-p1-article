% textidote: ignore begin
\section{Missing features from the program}\label{sec:improvements}
% textidote: ignore end

There are a few aspects of the program that the group would have liked to implement, but due to different circumstances
were not.
This section goes over the main features that were planned for the program, but were left out.
While these features are not required to make the program work, they were outlined in the analysis and design phase of
the project, so if the program were to be further developed, these features would be the first to be implemented.

\subsection{The Health Attribute}\label{subsec:health-attribute}

Initially, the four different attributes were price, time, environment and health.
When analyzing the problem, the complexity of calculating health benefits and consequences became apparent,
why this attribute was not implemented in the final version of the program.
Health is a wide topic, why it was of no use to somehow narrow it down to a single number to be evaluated upon.
Another idea was to hard code a value for different types of transport, but health is very user dependent.
Health can also be very subjective, as some people might find biking relaxing, while others might find it stressful.

\subsection{Cars and Bikes}\label{subsec:cars-bikes-walking}

Similar to the health attribute, we did not implement cars and bikes as a means of transportation.
This is unfortunate, but the reason behind the decision was the lack of time.
Our main focus was communicating with the Rejseplanen API, so we put our efforts into that.
Since it took longer than we expected, we decided to not implement cars and bikes.
However, that does not mean that it's the intended behavior.
The goal is to inform the user of the best route, regardless of the means of transportation.
So ideally the program should also show routes for cars and bikes.

It is worth noting however that there are already some implementations for using cars and bikes, such as giving the
user a choice of using them, saving them to the preferences file and already having the code to calculate the time,
price and environment impact of using them.

% textidote: ignore begin
\subsection{How we could improve the program}\label{sec:improve-program}
% textidote: ignore end

There are also a few ideas that the group thought of during or after the development of the program, that were not
implemented due to either time or complexity constraints.
As mentioned in the requirements, ideally the program would use a map API to properly calculate the distance and time
for a given route.
The map could also show a visual representation of any chosen route, which would be a good addition to the program.
While on the topic of visual representation, it would have been nice to have a graphical user interface, instead of
using the terminal.
The program could also benefit from having commuting specific features, such as notifying the user if there's a change
in their current commute route, such as planned track work, cancelled trains or other delays.
