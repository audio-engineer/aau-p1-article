\section{Evaluation}\label{sec:evaluation}

During and after the development of the program, we used different types of testing techniques to evaluate the accuracy
of the program.
Otherwise, we wouldn't know whether the program is working as intended.
During the development, we used CLion's built-in debugger to step through the program and evaluate the
values of the variables at each step.
We also used unit testing to test the individual components of the program, as described in
Section~\ref{subsec:testing}.
After the development, we used a qualitative evaluation method to determine the program's usability from the user's
perspective.
The method is called ``Thinking Aloud'', and the goal is to evaluate the program's UX by having the user say all their
thoughts out loud as they're using the program.
Below are our experiences and results with the different methods.

\subsection{Debugging}\label{subsec:debugging}

There were some cases during the development where we did not get the expected result from the program.
And in some extreme cases, the program could crash.
Due to the complexity of the program, it was not always easy to find the cause of the problem.
In these cases, we used the built-in debugger in CLion.
The way it works is that we can set breakpoints throughout the program, and when the program reaches a breakpoint, it
will pause the program and give us time to inspect and evaluate the values of the variables.
Then we can step through the program one step at a time and see how the values change.
If we notice anything unexpected, we can investigate further and find what the issue is.

An example of an issue we had was when we were testing Rejseplanens API.
Originally we were searching for a trip between ``Høje Taastrup St.'' and ``Sydhavn St.'', and we were getting the
expected result, but when we tried to search for a trip between ``Høje Taastrup St.'' and ``Slagelse St.'', we were
getting an error.
The problem seemed very strange, as it was giving an error only for some specific stations.
So we decided to use the debugger to look at where exactly the function was failing.
It was parsing the user input and returning the JSON response just fine, but when it tried to parse the JSON response,
the function would fail and return no results.
So we compared the two JSON responses, and we noticed that the only difference was that the station for ``Sydhavn St.''
required a change of line, while the station for ``Slagelse St.'' did not.
The JSON response has an element called ``Leg'', which would be an array if there were multiple legs (lines) in the
trip, and it would be an object if there was only one leg.
Our program was expecting an array, but it was getting an object, and that was causing the program to fail.

\subsection{Unit Testing}\label{subsec:unit-testing}

While working on different functions of the program, we didn't always have the opportunity to run them.
There could be many reasons for this, such as the function being dependent on other functions, or the function being
unneeded at the time.
But we still needed to test their functionality.
Therefore, we used unit testing to test the individual components of the program.
An example of this can be found when we were working on our calculation functions.
Since they were being worked on before we needed to use them, we couldn't test them by running the program.
So instead we would write a unit test in a C++ file, and then run the test to see if it passed or failed.
The way these tests work is that we call the function we want to test, and give it values to test.
We also have an expected result, and if the function returns the expected result, the test passes, otherwise it fails.

\begin{lstlisting}[label={lst:UnitTestExample}, caption={Example of a unit test.}]
  TEST(CalculatePriceTest, Train30kmAdult0) {
    const CalculatePriceParameters kParameters = {kTrain, 30, false, 0};
    EXPECT_EQ(900, CalculatePrice(&kParameters));
  }
\end{lstlisting}

Listing~\ref{lst:UnitTestExample} shows an example of a unit test that we use in our test suite.
It calls the function \texttt{CalculatePrice} with the parameters \texttt{kParameters}, and expects the result to be
900.
This one would check the price for a train trip of 30 kilometers, for an adult.
By manually calculating the price, we can set the expected result to 900.
If the function returns 900, the test passes.
If it returns anything else, we'll be notified, and we can see what the function actually returned.

\subsection{Calculation Evaluation}\label{subsec:calc-evaluation}

Speaking of the calculation functions, before developing them into the program we spent a considerable amount of time
on researching how to calculate the different attributes of a trip and the accuracy of these calculations.
For example, in the Section~\ref{subsec:calculating-price}, we would always make sure to check if our price results
would match with either the DSB price sheet~\cite{price_sheet} or the Rejseplanen price
calculator~\cite{price_calculator}.
Doing this assures the accuracy of the results, and we can be confident that the program is calculating the correct.
Of course, as we don't have access to enough data, we expect the results to be slightly off from the real world, but
we deem the margin of error to be reasonably small.

However, the same can't be said for the calculation of the duration of a trip.
We assume that the average speed in a city is 35 km/h~\cite{time_city}, and on the highway is 120
km/h~\cite{time_highway}.
But when comparing our results with the results from Google Maps, we found that our results were over 30\% slower.
The reason behind this is that Google Maps uses real time data to calculate the duration of a trip, while we use the
average speed.
One solution is to change the average speed to match the results from Google Maps.
While the result would again not be completely accurate, it would be much closer to the real world results.

% textidote: ignore begin
\subsection{Program Evaluation}\label{subsec:program-evaluation}
% textidote: ignore end

% TODO evaluate the program as a whole rather than the individual components

% textidote: ignore begin
\subsection{Thinking Aloud}\label{subsec:thinking-aloud}
% textidote: ignore end

% TODO Add user test
