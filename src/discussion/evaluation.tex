\section{Evaluation}\label{sec:evaluation}

During and after the development of the program, we used different types of testing techniques to evaluate the accuracy
of the program.
Otherwise, we wouldn't know whether the program is working as intended.
During the development, we used CLion's built-in debugger to step through the program and evaluate the
values of the variables at each step.
We also used unit testing to test the individual components of the program, as described in
Section~\ref{subsec:testing}.
After the development, we used a qualitative evaluation method to determine the program's usability from the user's
perspective.
The method is called ``Thinking Aloud'', and the goal is to evaluate the program's UX by having the user say all their
thoughts out loud as they're using the program.
Below are our experiences and results with the different methods.

\subsection{Debugging}\label{subsec:debugging}

There were some cases during the development where we did not get the expected result from the program.
And in some extreme cases, the program could crash.
Due to the complexity of the program, it was not always easy to find the cause of the problem.
In these cases, we used the built in debugger in CLion.
The way it works is that we can set breakpoints throughout the program, and when the program reaches a breakpoint, it
will pause the program and give us time to inspect and evaluate the values of the variables.
Then we can step through the program one step at a time and see how the values change.
If we notice anything unexpected, we can investigate further and find the cause of the problem.

An example of a problem we had was when we were testing Rejseplanens API.
Originally we were searching for a trip between ``Høje Taastrup St.'' and ``Sydhavn St.'', and we were getting the
expected result, but when we tried to search for a trip between ``Høje Taastrup St.'' and ``Slagelse St.'', we were
getting an error.
The problem seemed very strange, as it was giving an error only for some specific stations.
Therefore we used the debugger to look at where exactly the function was failing.
It was prasing the user input and returning the JSON response just fine, but when it tried to parse the JSON response,
the function would fail and return no results.
So we compared the two JSON responses, and we noticed that the only difference was that the station for ``Sydhavn St.''
required a change of line, while the station for ``Slagelse St.'' did not.
The JSON response has an element called ``Leg'', which would be an array if there were multiple legs (lines) in the
trip, and it would be an object if there was only one leg.
Our program was expecting an array, but it was getting an object, and that was causing the program to fail.

\subsection{Unit Testing}\label{subsec:unit-testing}

While working on different functions of the program, we didn't always have the opportunity to run them.
There could be many reasons for this, such as the function being dependent on other functions, or the function being
unneeded at the time.

\subsection{Thinking Aloud}\label{subsec:thinking-aloud}

