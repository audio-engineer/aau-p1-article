% textidote: ignore begin
\section{Review of Software Requirements}\label{sec:review-of-software-requirements}
% textidote: ignore end

We outlined some software requirement specifications in Section~\ref{sec:software-requirements-specification},
and in this section we will evaluate and review them.

\subsection{Hard requirements}\label{subsec:functional-requirements}

As expected, our ``need to have's'' are all implemented and functions as intended.
These are the requirements that the program wouldn't function without.
The program can parse user input, such as the user's desired start and end destination, and the time of departure or
arrival.
The program is also able to handle user inputs.

The program also successfully uses the Rejseplanen API to find trips between the user's desired start and end locations.
After the program receives the routes from the API, it can compare and evaluate each route according to the user's
preferences for each attribute.

However, we were not able to implement every attribute that we wanted.
Health was one of the attributes to be implemented when analyzing the problem, but for reasons discussed in
Section~\ref{subsec:health-attribute}, it was not implemented.
% TODO: write about output when finished

\subsection{Soft requirements}\label{subsec:non-functional-requirements}

We couldn't implement all of our ``nice to have's'', as some were out of the scope of our capabilities and of this
report.
While the program can search the Rejseplanen API for trips, we did not manage to implement the ability to select a
specific stop or station.
We could in theory create a user input that would allow the user to pick and choose a specific station from their search
, but as time was limited it was decided not to implement it.
Instead, the program selects the first result from the search.
This is not ideal, as it will sometimes select unintended stations.
However, due to time constraints, we were not able to implement a better solution.

It was considered to use a second API, like OpenStreetMap~\cite{openstreetmap} to make it easier to calculate the exact
distances and time between two locations, be it by foot, bike or car.
However, this would increase the complexity of the program, which is already sufficiently complex.
Therefore, we decided to do the calculations ourselves.

The project is cross-platform between Linux, macOS and Windows, as we are programming in C together and compiling the
code with the help of CMake, which enables the compilation of the program for all beforehand mentioned platforms.
Lastly, the program uses the terminal to interface with the user.
Our implementation is very simple, as we were not to implement a more complicated graphical user interface.
While the program can be compiled for mobile devices, the experience would probably not be very good, as the user
experience was focussed around PC users.

% textidote: ignore begin
\subsection{The Health Attribute}\label{subsec:health-attribute}
% textidote: ignore end

Initially, the four different attributes was price, time, environment and health.
When analyzing the problem, the complexity of calculating health benefits and consequences became apparent,
why this attribute was not implemented in the final version of the program.
Health is a wide topic, why it was of no use to somehow narrow it down to a single number to be evaluated upon.
Another idea was to hard code a value for different types of transport, but health is very user dependent.
Health can also be very subjective, as some people might find biking relaxing, while others might find it stressful.
