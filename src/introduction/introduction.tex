% textidote: ignore begin
\chapter{Introduction}\label{ch:introduction}
% textidote: ignore end

Commuting is a ubiquitous and integral part of modern life, affecting millions of people daily as they travel between
their residences and workplaces, educational institutions, or other destinations.
It provides a steady and reliable transportation method for many individuals on a daily basis.
Many individuals travel by the mercy of the default configurations of existing commuting software, without realizing the
potential of the personalized commuting experience.

However, to commute can have its consequences, hereby impacts on ones well-being and the climate as well as other
factors.
Climate especially is a very hot topic which we believe many commuters lack information about, and hereby are not able
to make sufficiently informed decisions around their commuting habits.

Relying on existing public transport pathfinding software, we will explore a gap in the services provided by these,
being the possibility to tailor the commuting experience.
Therefore, in this paper we will explore what factors could be interesting for user to be able to tweak to provide them
with the best route suited for their preferences within the always current status quo of the public transport
availability.
