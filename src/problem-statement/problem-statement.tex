% TODO Make chapter much longer and remove TeXtidote ignore annotation
% textidote: ignore begin
\chapter{Problem statement}\label{ch:problem-statement}
% textidote: ignore end

The following chapter entails the problem statement to be concluded upon.

% textidote: ignore begin
\section{Problem delineation}\label{sec:problem-delineation}
% textidote: ignore end

In contemporary society, the pursuit of employment in metropolitan areas clashes with the desire to maintain close
familial ties in hometowns.
As a result, individuals have to make the complex decision of whether commuting should be chosen as the means of
going to and from work.

The intricate considerations encompassing transportation options, time allocation between work and family, environmental
implications, and work flexibility create a challenging landscape for prospective as well as established commuters.

% textidote: ignore begin
\section{Problem statement}\label{sec:problem-statement}
% textidote: ignore end

% textidote: ignore begin
How can we create software to help commuters choose a commuting route that best suits their preferences?
% textidote: ignore end

% textidote: ignore begin
\section{Sub questions to problem statement}\label{sec:sub-questions-to-problem-statement}
% textidote: ignore end

% textidote: ignore begin
To answer this question, we will also be asking ourselves some sub questions that align with the overall problem
statement to conclude with a more nuanced answer.
% textidote: ignore end

\begin{itemize}
    \item What can we do to inform commuters about the environmental impact that different forms of transportation
    cause in their commuting routes?
    \item How can we help individuals make a choice on whether to move closer to their place of occupation or to commute
    from their current place of living?
\end{itemize}
