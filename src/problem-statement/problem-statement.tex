% TODO Make chapter much longer and remove TeXtidote ignore annotation
% textidote: ignore begin
\chapter{Problem statement}\label{ch:problem-statement}
% textidote: ignore end

Now we want to define the problem statement.


\section{Statement}\label{sec:statement}

In contemporary society, the pursuit of employment in metropolitan areas clashes with the desire to maintain close
familial ties in hometowns.
As a result, individuals have to make the complex decision of whether commuting should be chosen as the means of
going to and from work.

The intricate considerations encompassing transportation options, time allocation between work and family, environmental
implications, and work flexibility create a challenging landscape for prospective commuters.

We believe that the issue is not that not enough people take these metrics into account.
We believe that the reason for these challenges is the complexity of obtaining high-quality information, thus reducing
awareness and a base for a proper decision.
By giving potential commuters an app to plan their route according to their preferences, many more would try commuting.

This project attempts to develop a decision-support system tailored to aid individuals struggling with the choice of
commuting, aiming to integrate diverse commuting strategies into a compact and efficient application solution.

By addressing the multifaceted nature of commuting decisions, our solution seeks to optimize commuting choices,
considering cost-effectiveness, time efficiency, and environmental impact, thus assisting individuals in making
informed and balanced decisions about their commuting commitments.
